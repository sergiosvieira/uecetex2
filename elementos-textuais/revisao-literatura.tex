\section{Revisão de Literatura}
\label{sec:revisao}

\subsection{Enfermagem na APS - Os atributos da Atenção Primária e a Enfermagem na ESF}

A história da saúde brasileira caminhou junto com o desenvolvimento da sociedade e da lutas por direitos sociais. Após muitas disputas e conquistas, surgiu o \acrshort{SUS} que traz como princípios básicos a integralidade, a universalidade e a equidade. 

A assistência dentro do \acrshort{SUS} passou a se constituir em níveis de atenção primário, secundário e terciário. Na \acrshort{APS}, que no Brasil denominou-se \acrlong{ABS} pela necessidade de desconstruir a ideia de uma assistência sem conhecimentos complexos voltados para a população de baixa renda, tem lugar a \acrlong{ESF} que visa, reorientar o modelo assistencial para um universal que engloba os diferentes setores \cite{oliveira2013atributos}.

A enfermagem tem importante papel dentro da \acrshort{ESF}, pois de acordo com os estudo de \cite{fernandes2013gerencia}, as ações de cuidado e gerência do enfermeiro na \acrshort{ESF}, busca construir novas práticas, acompanhando inclusive a proposta de reorientação da assistência à saúde do \acrshort{SUS}, mas muitas vezes ainda permanecem orientadas nos moldes tradicionais da assistência. 

Como um resgate às propostas orientadas pela Reforma Sanitária, a \acrshort{ABS} traz com ela derivações dos atributos da \acrshort{APS} a saber: a atenção no primeiro contato, a longitudinalidade, a integralidade e a coordenação \cite{oliveira2013atributos}.

A atenção no primeiro contato trata do acesso ao serviço de saúde, de onde o usuário sente-se mais impelido a buscar o auxílio de que necessita. Como diz \cite{oliveira2013atributos}, ``o primeiro contato pode ser definido como porta de entrada dos serviços de saúde, ou seja, quando a população e a equipe identificam aquele serviço como o primeiro recurso a ser buscado quando há uma necessidade ou problema de saúde.''

A longitudinalidade como termo, ainda não é usual em nossa literatura, pois no estudo de \cite{cunha2011longitudinalidade}, ele trata da proximidade com o termo continuidade do cuidado. Mesmo assim, aponta para a ideia do vínculo que se estabelece, dando ênfase ao acolhimento que ainda segundo as autoras, é uma tecnologia que reformula processos de trabalho e relacionamentos  entre profissionais e usuários, sendo usada para alcançar a longitudinalidade. ``Está fortemente relacionada à boa comunicação e tende a favorecer a continuidade e a efetividade do cuidado, contribuindo para a implementação de ações de promoção e de prevenção de agravos.'' \cite{cunha2011longitudinalidade}. 

Outro atributo é a integralidade, já presente como princípio do \acrshort{SUS}. Para \cite{oliveira2013atributos} a integralidade depende de várias estratégias como redefinição das práticas, da forma de criar vínculos, acolhimento e autonomia. Isso valoriza a subjetividade dos trabalhadores na saúde e as singularidades dos sujeitos, com isso abrindo espaço para a construção de um cuidado realmente centrado no usuário. 

O último atributo é a coordenação que aborda a articulação das ações e serviços de saúde entre os níveis assistenciais. Assim, alcança-se uma integralidade da assistência através do funcionamento bem articulado de toda a rede de atenção à saúde \cite{oliveira2013atributos}.

Percebe-se assim que o trabalho da enfermagem dentro da \acrshort{ESF} tem a possibilidade de alinhamento com seus atributos pois, seja na assistência direta ou indireta, o enfermeiro acaba por exercê-los para uma boa prática, articulando-se ainda com outros serviços para resolver as demandas.

\begin{citacao}
O papel do enfermeiro é reconhecido, em suma, pela capacidade e habilidade de compreender o ser humano como um todo, pela integralidade da assistência à saúde, pela capacidade de acolher e identificar-se com as necessidades e expectativas dos indivíduos e famílias, pela capacidade de acolher e compreender as diferenças sociais, bem como pela capacidade de promover a interação e associação entre os usuários, a equipe de saúde da família e a comunidade. A enfermagem se aproxima, identifica e procura criar uma relação efetiva com o usuário, independentemente das suas condições econômicas, culturais ou sociais, ou seja, busca otimizar as intervenções de cuidado em saúde de modo que integre e contemple tanto os saberes profissionais quanto os saberes dos usuários e da comunidade. \cite[p. 568]{backes2015significado}
\end{citacao}

Ao alinhar suas ações e competências a essa forma de cuidado, o enfermeiro acaba por estabelecer contatos que formarão redes nas quais ele pode ser ponto forte ou não. Além disso, abre-se a possibilidade de se explorar essas redes de forma a se conhecer em maior profundidade as relações dos indivíduos com suas redes.  

\subsection{Apoio social e Redes Sociais - transitando pelos territórios de produção do cuidado}

A comunicação entre os diferentes atores sociais geram fluxos de informações, sentimentos e trocas que quando encontram significado entre esses atores, acabam por formar vínculos, laços, e conexões, formando uma rede.  

\begin{citacao}
Redes sociais são teias de relações e trocas de obrigações postas pela organização social e cultura e não somente elos entre indivíduos favorecidos somente pelos vínculos e ligações afetivas entre eles. Não são recursos abstratos mobilizados como apoio ou ajuda no cuidado e na proteção à saúde, embora sejam acionados e ofertados, circunstancialmente. \cite[p. 1109]{canesqui2012apoio}
\end{citacao}

Então, as redes podem surgir mesmo em meio a indivíduos que não sejam do mesmo convívio diário, não se limitando apenas a membros da família ou amigos, embora esses possam também formá-las. Assim, ``as redes aplicam-se aos pequenos grupos, ao sistema global e de comunicação, cujas relações estabelecidas diferenciam-se em simétricas ou as simétricas; nas suas direções e quanto à existência ou não de reciprocidade e troca.'' \cite{canesqui2012apoio}

Dentro dessas redes um sujeito pode se sentir amparado e ser ou ter referência entre os diferentes indivíduos que as compõem. Surge então a ideia de apoio social para aquele sujeito dentro da rede. Uma ideia que para \cite{canesqui2012apoio} diz que:

\begin{citacao}
Sob as abordagens simultâneas, macro e micro analíticas, o apoio social pode ser visto como um tipo de prestação de ajuda que repousa, de um lado nos intercâmbios, obrigações e padrões de reciprocidade entre indivíduos, grupos, famílias e instituições, portando significados para os atores neles envoltos, nas suas respectivas experiência cotidianas e contextos. \cite[p. 1104]{canesqui2012apoio}
\end{citacao}

No que concerne à saúde, o apoio social nas redes sociais tem importância, pois dependendo de como é a força dessa rede e sua extensão, tem-se também um grande número de possibilidades de ações de cuidado. É o que trazem  \cite{canesqui2012apoio}:

\begin{citacao}
``Estudos associam o apoio social ao tamanho das redes sociais, à integração social, ao desempenho de papéis sociais e às demandas de cuidados aos doentes. Nas redes sociais constrangidas e restritas, as pessoas possuem menores chances de receber apoios, na condição de pobreza e de isolamento social'' \cite[p. 1108]{canesqui2012apoio}
\end{citacao}

\subsection{Redes sociais articuladas no SUS e nas equipes a partir do enfermeiro}

A rede social pode ser usada como uma estratégia para o cuidado integral, uma vez que nas relações nela existentes dão-se as trocas e é onde o indivíduo pode se sentir amparado em suas necessidades. A Enfermagem dentro do SUS, sobretudo na ABS, foca na atenção também às famílias, podendo estudá-las 

Rede como estratégia no cuidado integral
Enfermeiro como articulador, apoio, elo fortemente
Instrumentos utilizados pela enfermagem para traçar as redes de apoio
(Enfermagem tem se apropriado de algumas tecnologias para traçar essas redes- ecomapa
rede de ecology (practice))
