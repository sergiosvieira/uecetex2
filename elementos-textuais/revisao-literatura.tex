\section{Revisão de Literatura}
\label{sec:revisao}

\subsection{A Sífilis na Gestação: um problema simples e longe de resolução}
A sífilis, por ser quase que exclusivamente transmitida pelo contato sexual, é considerada uma infecção sexualmente transmissível, pandêmica que inicia seu ciclo por meio de cancros, evoluindo posteriormente para sua forma crônica. Pode afetar diversos sistemas do organismo e sem tratamento adequado, pode levar a óbito. Em alguns casos ela pode infectar o feto por transmissão vertical via transplacentária, ocasionando a sífilis congênita. \cite{silva2015feelings}

É também um importante problema de saúde pública, pois além de seu potencial infectocontagioso, ela pode debilitar de modo grave o organismo e quando não tratada, aumenta o risco para infecção do HIV por meio de suas lesões que servem como porta de entrada. \cite{manual2016sifilis}

Sem tratamento, a sífilis progride apresentando três fases sintomáticas, permeadas por períodos assintomáticos (fases latentes). Determinar o estágio em que ela se apresenta é importante para o método diagnóstico. Entretanto é uma doença difícil de diagnosticar, justamente por ter períodos sem manifestações de sintomas, além de período de incubação variável. \cite{manual2016sifilis}

A sífilis primária apresenta úlcera única, endurecida e não dolorosa  na região genital, o cancro duro. Essa lesão pode desaparecer naturalmente, mesmo sem tratamento. Já a sífilis secundária, se apresenta entre seis e oito semanas após a primeira lesão podendo manifestar lesões cutâneas e nas mucosas, sendo comum as roséolas. O indivíduo pode ou não ter febre, mal-estar, fraqueza muscular e cefaléia. Aqui as lesões são mais visíveis, mas também desaparecem. A sífilis terciária só aparece 3 a 12 anos depois do contágio com lesões neurológicas e complicações mais sérias: problemas ósseos e cardiovasculares. Os períodos de latência não apresentam sintomas. São diferenciados em recente (menos de um ano) e tardia (mais de uma ano). Nesse caso a doença só é detectada por meio de testes sorológicos. \cite{manual2016sifilis}

Quanto às gestantes, se não tratadas adequadamente, podem ocasionar a transmissão vertical, gerando a sífilis congênita. Essa transmissão se torna mais provável quanto mais recente for a infecção. 

No período gestacional o diagnóstico e tratamento ocorrem durante as consultas de  pré-natal. Todas as gestantes e suas parcerias sexuais precisam ser investigadas  e também informadas sobre as formas de contágio e riscos para a gestação e o bebê. Pelo protocolo do Ministério da Saúde, o uso de testes-rápidos treponêmicos   servem como triagem para gestantes e seus parceiros. Embora essa seja uma intervenção eficaz,  depende de muitos fatores para ser efetiva. Já o VDRL é solicitado logo na primeira consulta de pré-natal, de preferência ainda no primeiro trimestre gestacional. O teste rápido também é realizado no início do terceiro trimestre e no momento do parto. Além disso, deve ser realizado também em casos de abortamento. \cite{brasil2015protocolo}

Para um diagnóstico mais completo incluindo o estadiamento da infecção, usam-se testes treponêmicos e o VDRL, pois ambos são complementares, uma vez que juntos, diminuem o erro por conta de falsos positivos e negativos. Os testes treponêmicos identificam anticorpos específicos para o Treponema pallidum, agente etiológico da sífilis, sendo usado na triagem ou confirmação de resultado. \cite{sao2016guia}

Toda gestante diagnosticada com sífilis deve iniciar o tratamento imediatamente. A única forma de tratamento efetivo é o uso da penicilina G benzatina. O parceiro também é procurado para teste rápido e posterior tratamento. Em caso de sensibilização da gestante à penicilina, inicia-se tentativa de dessensibilização. Caso seja impossível o processo, o tratamento é feito com ceftriaxona, entretanto é considerado um tratamento inadequado e neste caso, o recém-nascido será avaliado clínica e laboratorialmente. \cite{sao2016guia}

O tratamento na gestante é considerado adequado quando: é completo, documentado e adequado ao estágio da doença, sendo realizado com penicilina G benzatina; o parceiro também é tratado; a gestante apresenta queda em suas titulações para os testes sorológicos não-treponêmicos, ou estáveis se os títulos forem menor ou igual que 1:4. \cite{sao2016guia}

É possível observar através de manuais frequentemente elaborados que os processos para rastreamento, diagnóstico e tratamento da gestante estão bem estabelecidos e institucionalizados transversalmente pelo Ministério da Saúde por meio de programas, estratégias e políticas. Mesmo assim, a sífilis na gestação e também a congênita, continuam a ser um problema longe de ser solucionado, pois dependem de diversos fatores ligados não apenas a assistência e a gestão, como também a cultura, nível socioeconômico e outros. 

Para \cite{cad2013} a sífilis em gestantes se relaciona ao baixo nível socioeconômico. Mesmo não se restringindo à população menos favorecida, os resultados de seu estudo indicam que pouca escolaridade e baixa renda podem ser sinais importantes de acesso limitado aos serviços de  saúde.

Até mesmo no que se refere às notificações de casos, surgem falhas nesse processo, o que leva a um mal direcionamento das políticas públicas para sua prevenção e controle. 

Para \cite{domingues2016incidencia},  a quantidade de casos notificados depende da capacidade de intervenção direcionada a redução da transmissão vertical; diagnóstico e tratamento bem executados tanto das gestantes quanto de seus parceiros e da notificação correta, pois números baixos de casos de sífilis não garantem que a assistência está sendo eficiente no controle da doença, pois pode estar havendo subnotificação. Ao mesmo tempo, um alto número de casos indicam falhas na assistência.

O controle da sífilis e de de outras IST no Brasil precisa ser um processo dinâmico, sendo renovado constantemente, necessitando de protagonismo por parte dos trabalhadores da saúde, além das responsabilidades de cada órgão que compõe o SUS. É fundamental a existência de saberes e práticas de gestão para que as políticas possam ser estruturadas e executadas de acordo com os princípios do SUS. \cite{brasil2015protocolo}

