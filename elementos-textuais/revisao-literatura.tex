\chapter{Reverêncial Teórico}
\label{sec:revisao}

\section{A violência obstétrica e a enfermagem}

Diante de toda leitura realizada para embasar o estudo, nesta seção ratifica-se o que há escrito sobre o tema escolhido que é: Percepções do enfermeiro sobre a violência obstétrica no Centro de Parto Normal.

A violência pode ser definida como a imposição da força causando um grau significativo de dor e sofrimentos que podem ser evitáveis; especificamente, a violência obstétrica contra a mulher abrange a violência física, sexual e/ou psicológica. \cite{cardoso2017violencia}.

Dentre os tipos de violência, encontra-se a violência obstétrica, caracterizada pelas variadas formas de violência ocorridas na assistência à gravidez, ao parto, ao pós-parto e ao abortamento e que está cada vez mais empregado pelo ativismo social, em pesquisas acadêmicas e na elaboração de políticas públicas. É caracterizada como o agrupamento de formas de violência e danos originados no cuidado obstétrico profissional, sendo recentemente reconhecida como questão de saúde pública pela OMS \cite{diniz2015abuse,sena2016violencia}.

Observa-se na prática de profissionais da Obstetrícia, sejam médicos, enfermeiros, enfermeiros obstetras ou técnicos de enfermagem, o despreparo, a imperícia e a negligência de informações, emoções, sentimentos, percepções e direitos da mulher no gestar e parir, sendo impedidas de decidir a posição que querem ter seus filhos, de expressar seus sentimentos e emoções, de ter a presença de acompanhantes, infringindo a Política Nacional de Humanização e mudando o foco da mulher para o procedimento, deixando-as mais vulneráveis à violência e menos empoderadas do seu corpo, silenciada pelos profissionais e pela própria parturiente \cite{andrade2014violencia,gonccalves2014violencia}


\citeauthor{diniz2015abuse} (\citeyear{diniz2015abuse}), em seu artigo descreve algumas categorias de desrespeito e abuso às parturientes, como: abuso físico, imposição de intervenções não consentidas; intervenções aceitas com base em informações parciais ou distorcidas; cuidado não confidencial ou não reservado; cuidado indigno e abuso verbal; discriminação baseada em certos atributos; abandono, negligência ou recusa de assistência; detenção nos serviços. Além disso, cita alguns exemplos de violência obstétrica, relacionadas a estes desrespeitos e abusos: procedimentos sem justificativa clínica e intervenções ``didáticas'', como toques vaginais dolorosos e repetitivos, cesáreas e episiotomias desnecessárias, imobilização física em posições dolorosas, prática da episiotomia e outras intervenções sem anestesia, sob a crença de que a paciente ``já está sentindo dor mesmo''; desrespeito ou desconsideração do plano de parto; indução à cesárea por motivos duvidosos; maternidades que possuem enfermarias coletivas sem separação de biombo; tratamento diferenciado devido à classe social, raça e entres outros agindo de preconceito.

No Estado de Santa Catarina, em 17 de janeiro de 2017, tivemos um avanço significativo no reconhecimento da violência obstétrica com a Lei n. 17.097, que dispõe sobre a implantação de medidas de informação e proteção à gestante e parturiente contra a violência obstétrica no estado e divulgação da Política Nacional de Atenção Obstétrica Neonatal. A Lei classifica o que é violência obstétrica em vários níveis e a fiscalização das redes de saúde pelos órgãos públicos, que serão responsáveis também pela aplicação das sanções decorrentes de infrações às normas nela contidas, mediante processo administrativo, assegurada ampla defesa. \cite{soares2017violencia}.

\citeauthor{soares2017violencia} (\citeyear{soares2017violencia}), afirma em seu artigo que mesmo com todas as informações, estudos, discussões e implementações sobre o tema, nota-se a necessidade da elaboração de uma lei federal para combater diretamente a violência obstétrica em todos os estados do Brasil, na qual elabore fiscalização e punição aos agressores, tendo em vista a proteção e o bem-estar da mulher e do bebê. 

\begin{citacao}
Ao final do mês de maio de 2014, foi protocolado o Projeto de Lei n. 7.633, assinado pelo Deputado Jean Wyllys, com a iniciativa de garantir a gestante o direito ao parto e abortamento humanizado, a publicidade de informações sobre o direito ao parto humanizado nos estabelecimentos de saúde, além de propor, entre outras coisas, que os profissionais de saúde que praticarem a violência obstétrica fiquem sujeitos à responsabilização civil e criminal. \cite{soares2017violencia}.
\end{citacao}

Diante do que foi exposto, das diversas formas de violência obstétrica, das suas consequências e alta prevalência, para a solução desse problema faz-se fundamental a adesão às mudanças no modelo da assistência obstétrica com o caráter mais humanizado, incluindo a mudança da grade curricular da formação dos profissionais da saúde como: Médicos Obstetras, Enfermeiros Obstetras;  pactuação das instâncias da rede de cuidados, co-responsabilização do cuidado e respeito aos direitos da mulher \cite{rodrigues2018violencia,gonccalves2014violencia}.

\section{O equipamento de saúde Centro de Parto Normal}

O movimento feminista foi de grande importância para implantação do Centro de Parto Normal (\acrshort{CPN}) na década de 1980, pois passou a questionar as práticas obstétricas de rotina e repensar formas para realizar uma assistência ao parto e nascimento mais humanizado. Mas mesmo a regulamentação do CPN em 1999, o número de estabelecimentos no Brasil ainda é reduzido. Somos conhecidos mundialmente pela elevada incidência de cesarianas, com taxas maiores de 50\%. \cite{garcia2017centro}. ``O CPN é um equipamento de cuidado para a redução das taxas de cesáreas, pois possibilita a diminuição das intervenções obstétricas'' \cite{osava2011caracterizaccao}. 

O CPN é uma unidade de assistência ao parto de risco habitual sem distócia, ou seja, sem complicações obstétricas. Nesta perspectiva, a assistência no CPN dispõe de um conjunto de elementos destinados a receber a parturiente e seu acompanhante permitindo um trabalho de parto ativo, humanizado e participativo, caracterizando o uso das boas práticas, diferenciando-se, assim, dos serviços tradicionais de atenção obstétrica, que é o modelo mais intervencionista. \cite{garcia2017centro}.

A especialidade em obstetrícia pode ser exercida tanto por médicos quanto por enfermeiros. Mas a ``arte de partejar'' nunca foi objeto de monopólio médico, pois quem dominava essa arte era as parteiras e a enfermagem. \cite[p. 69]{maia2010humanizaccao}.

A trajetória da enfermagem obstétrica na atenção à parturiente e ao parto normal é longa. Sua consolidação ocorreu mediante a lei n. 7498/86 e o decreto que a regulamentou, de n. 94.406/87, segundo os quais cabe a enfermeira obstetra o compromisso de assistir a parturiente e o parto normal. Em meados dos anos 1990, muitos desses profissionais especialistas incorporaram, em seu fazer, práticas obstétricas recomendadas pela OMS e consideradas apropriadas pelo Ministério da Saúde (\acrshort{MS}). ``Assim, a enfermeira obstetra agregou conhecimentos técnicos a uma atenção humanizada e de qualidade, respeitando os preceitos éticos e garantindo a privacidade e autonomia da mulher.'' \cite{vargens2017contribuiccao}. 

A inserção da enfermeira obstetra tem o objetivo de qualificar a assistência obstétrica por meio do uso de métodos não farmacológicos e não invasivos, a fim de diminuir o número de intervenções medicamentosas e práticas obstétricas indevidas, estimular o trabalho de parto, além de favorecer a fisiologia do corpo da mulher. 
\cite[p.21]{maia2010humanizaccao,vico2017avaliaccao}.

De acordo com a Resolução n. 516/2016 do Conselho Nacional de Enfermagem (\acrshort{COFEN}), o CPN e/ou Casa de Parto destinam-se à assistência ao parto e nascimento de baixo risco, conduzido pela(o) Enfermeira(o), Enfermeira(o) Obstetra ou Obstetriz, da admissão até a alta. No CPN a enfermeira obstetra tem papel central no atendimento a gestante de risco habitual e ao recém-nascido, sendo necessário chamar o profissional médico somente quando essa parturiente passa para médio/alto risco. \cite{deresoluccao}.

Compete ao Enfermeira(o), Enfermeira(o) Obstetra ou Obstetriz que atua no serviço de obstetrícia, CPN e/ou Casa de Parto ou outro local onde ocorra a assistência: Emissão de laudos de autorização de Internação Hospitalar (\acrshort{AIH}) para o procedimento de parto normal sem distócia, realizado pelo Enfermeiro(a) Obstetra, da tabela do \acrshort{SIH/SUS}; Identificação das distócias obstétricas e tomada de providências necessárias, até a chegada do médico, devendo intervir, em conformidade com sua capacitação técnico-científica, adotando os procedimentos que entender imprescindíveis, para garantir a segurança da mãe e do recém-nascido; Realização de episiotomia e episiorrafia (rafias de lacerações de primeiro e segundo grau) e aplicação de anestesia local, quando necessária; Acompanhamento obstétrico da mulher e do recém-nascido, sob seus cuidados, da internação até a alta. \cite{deresoluccao}.

Acredita-se que o conhecimento sobre as contribuições a cerca dessa ferramenta de cuidado, o CPN, permitirá uma maior divulgação das possíveis contribuições desta estratégia de humanização ao parto. A disseminação deste conhecimento auxiliará a enfermagem e os profissionais de saúde a desenvolver ações humanizadas e de acordo com as políticas de saúde vigentes. \cite{garcia2017centro}.