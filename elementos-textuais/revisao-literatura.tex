\section{Revisão de Literatura}
\label{sec:revisao}

\subsection{Enfermagem na APS - Os atributos da Atenção Primária e a Enfermagem na ESF}

A história da saúde brasileira caminhou junto com o desenvolvimento da sociedade e da lutas por direitos sociais. Após muitas disputas e conquistas, surgiu o \acrshort{SUS} que traz como princípios básicos a integralidade, a universalidade e a equidade. 

A assistência dentro do \acrshort{SUS} passou a se constituir em níveis de atenção primário, secundário e terciário. Na \acrshort{APS}, que no Brasil denominou-se \acrlong{ABS} pela necessidade de desconstruir a ideia de uma assistência sem conhecimentos complexos voltados para a população de baixa renda, tem lugar a \acrlong{ESF} que visa, reorientar o modelo assistencial para um universal que engloba os diferentes setores \cite{oliveira2013atributos}.

A enfermagem tem importante papel dentro da \acrshort{ESF}, pois de acordo com os estudo de \cite{fernandes2013gerencia}, as ações de cuidado e gerência do enfermeiro na \acrshort{ESF}, busca construir novas práticas, acompanhando inclusive a proposta de reorientação da assistência à saúde do \acrshort{SUS}, mas muitas vezes ainda permanecem orientadas nos moldes tradicionais da assistência. 

Como um resgate às propostas orientadas pela Reforma Sanitária, a \acrshort{ABS} traz com ela derivações dos atributos da \acrshort{APS} a saber: a atenção no primeiro contato, a longitudinalidade, a integralidade e a coordenação \cite{oliveira2013atributos}.

A atenção no primeiro contato trata do acesso ao serviço de saúde, de onde o usuário sente-se mais impelido a buscar o auxílio de que necessita. Como diz \cite{oliveira2013atributos}, ``o primeiro contato pode ser definido como porta de entrada dos serviços de saúde, ou seja, quando a população e a equipe identificam aquele serviço como o primeiro recurso a ser buscado quando há uma necessidade ou problema de saúde.''

A longitudinalidade como termo, ainda não é usual em nossa literatura, pois no estudo de \cite{cunha2011longitudinalidade}, ele trata da proximidade com o termo continuidade do cuidado. Mesmo assim, aponta para a ideia do vínculo que se estabelece, dando ênfase ao acolhimento que ainda segundo as autoras, é uma tecnologia que reformula processos de trabalho e relacionamentos  entre profissionais e usuários, sendo usada para alcançar a longitudinalidade. ``Está fortemente relacionada à boa comunicação e tende a favorecer a continuidade e a efetividade do cuidado, contribuindo para a implementação de ações de promoção e de prevenção de agravos.'' \cite{cunha2011longitudinalidade}. 

Outro atributo é a integralidade, já presente como princípio do \acrshort{SUS}. Para \cite{oliveira2013atributos} a integralidade depende de várias estratégias como redefinição das práticas, da forma de criar vínculos, acolhimento e autonomia. Isso valoriza a subjetividade dos trabalhadores na saúde e as singularidades dos sujeitos, com isso abrindo espaço para a construção de um cuidado realmente centrado no usuário. 

O último atributo é a coordenação que aborda a articulação das ações e serviços de saúde entre os níveis assistenciais. Assim, alcança-se uma integralidade da assistência através do funcionamento bem articulado de toda a rede de atenção à saúde \cite{oliveira2013atributos}.

Percebe-se assim que o trabalho da enfermagem dentro da \acrshort{ESF} tem a possibilidade de alinhamento com seus atributos pois, seja na assistência direta ou indireta, o enfermeiro acaba por exercê-los para uma boa prática, articulando-se ainda com outros serviços para resolver as demandas.

\begin{citacao}
O papel do enfermeiro é reconhecido, em suma, pela capacidade e habilidade de compreender o ser humano como um todo, pela integralidade da assistência à saúde, pela capacidade de acolher e identificar-se com as necessidades e expectativas dos indivíduos e famílias, pela capacidade de acolher e compreender as diferenças sociais, bem como pela capacidade de promover a interação e associação entre os usuários, a equipe de saúde da família e a comunidade. A enfermagem se aproxima, identifica e procura criar uma relação efetiva com o usuário, independentemente das suas condições econômicas, culturais ou sociais, ou seja, busca otimizar as intervenções de cuidado em saúde de modo que integre e contemple tanto os saberes profissionais quanto os saberes dos usuários e da comunidade. \cite[p. 568]{backes2015significado}
\end{citacao}

Ao alinhar suas ações e competências a essa forma de cuidado, o enfermeiro acaba por estabelecer contatos que formarão redes nas quais ele pode ser ponto forte ou não. Além disso, abre-se a possibilidade de se explorar essas redes de forma a se conhecer em maior profundidade as relações dos indivíduos com suas redes.  

\subsection{Apoio social e Redes Sociais - transitando pelos territórios de produção do cuidado}

A comunicação entre os diferentes atores sociais geram fluxos de informações, sentimentos e trocas que quando encontram significado entre esses atores, acabam por formar vínculos, laços, e conexões, formando uma rede.  

\begin{citacao}
Redes sociais são teias de relações e trocas de obrigações postas pela organização social e cultura e não somente elos entre indivíduos favorecidos somente pelos vínculos e ligações afetivas entre eles. Não são recursos abstratos mobilizados como apoio ou ajuda no cuidado e na proteção à saúde, embora sejam acionados e ofertados, circunstancialmente. \cite[p. 1109]{canesqui2012apoio}
\end{citacao}

Então, as redes podem surgir mesmo em meio a indivíduos que não sejam do mesmo convívio diário, não se limitando apenas a membros da família ou amigos, embora esses possam também formá-las. Assim, ``as redes aplicam-se aos pequenos grupos, ao sistema global e de comunicação, cujas relações estabelecidas diferenciam-se em simétricas ou as simétricas; nas suas direções e quanto à existência ou não de reciprocidade e troca.'' \cite{canesqui2012apoio}

Dentro dessas redes um sujeito pode se sentir amparado e ser ou ter referência entre os diferentes indivíduos que as compõem. Surge então a ideia de apoio social para aquele sujeito dentro da rede. Uma ideia que para \cite{canesqui2012apoio} diz que:

\begin{citacao}
Sob as abordagens simultâneas, macro e micro analíticas, o apoio social pode ser visto como um tipo de prestação de ajuda que repousa, de um lado nos intercâmbios, obrigações e padrões de reciprocidade entre indivíduos, grupos, famílias e instituições, portando significados para os atores neles envoltos, nas suas respectivas experiência cotidianas e contextos. \cite[p. 1104]{canesqui2012apoio}
\end{citacao}

No que concerne à saúde, o apoio social nas redes sociais tem importância, pois dependendo de como é a força dessa rede e sua extensão, tem-se também um grande número de possibilidades de ações de cuidado. É o que trazem  \cite{canesqui2012apoio}:

\begin{citacao}
``Estudos associam o apoio social ao tamanho das redes sociais, à integração social, ao desempenho de papéis sociais e às demandas de cuidados aos doentes. Nas redes sociais constrangidas e restritas, as pessoas possuem menores chances de receber apoios, na condição de pobreza e de isolamento social'' \cite[p. 1108]{canesqui2012apoio}
\end{citacao}

\subsection{Redes sociais articuladas na ESF e os instrumentos de análise pelo Enfermeiro}
A rede social pode ser usada como uma estratégia para o cuidado integral, uma vez que nas relações nela existentes dão-se as trocas entre os sujeitos. Além disso é onde o indivíduo pode se sentir amparado em suas necessidades. 

As redes sociais apresentam expressiva importância, na medida em que influenciam a auto-imagem do indivíduo e são centrais para a experiência de identidade e competência, muito particularmente na atenção à saúde e adaptação em situações de crise. \cite{lima2012rede}

A Enfermagem na \acrshort{ABS}, sobretudo na \acrshort{ESF}, foca sua prática também às famílias, podendo estudá-las através de suas relações. Para isso ela faz uso de instrumentos que possibilitam conhecer o papel dos indivíduos e as relações de troca (ou não) entre eles. O resultado dessa avaliação, permite uma leitura da situação familiar, suscitando ações e intervenções para a promoção de saúde daqueles sujeitos. Para \cite{gisele2013book}.

\begin{citacao}
Na \acrshort{ESF}, a utilização regular desses instrumentos facilita a determinação das principais características e peculiaridades familiares, bem como seus pontos fortes e suas fragilidades, facilitando a atuação, a priorização do atendimento e cuidado, o registro e a discussão, com os demais profissionais da equipe, sobre cada unidade familiar. \cite{gisele2013book}
\end{citacao}

Dentre os instrumentos, considera-se o ecomapa uma ferramenta adequada, pois para \cite[p. 323]{lima2012rede}, ele permite um cuidado individualizado a partir do delineamento da rede social e de apoio de um indivíduo. Ele proporciona a percepção desse apoio recebido e utilizado pelo indivíduo, se ele é recíproco, que o proporciona e a orientação dele dentro da rede. 

Investigando o papel da família como rede de apoio o enfermeiro também pode fazer uso do genograma, pois ele oferece um desenho estrutura relacional da família, trazendo dados que ``denotam a estrutura da família e podem se configurar como indícios do funcionamento e dinâmica das mesmas.'' \cite{wendt2008utilizaccao}

\cite{gisele2013book} colocam que:

\begin{citacao}
No genograma, os membros da família e as diversas gerações são representadas graficamente e visualizados por meio de símbolos padronizados. Constitui-se em uma árvore genealógica que detalha a estrutura interna da família, ao fornecer informações sobre os vários papéis de seus membros atuais e de outras gerações, possibilitando a discussão e análise das interações existentes entre eles. \cite{gisele2013book}
\end{citacao}

Ainda de acordo com as autoras, ``tanto o genograma quanto o ecomapa, exige um estabelecimento de uma interação efetiva com as famílias.'' \cite{gisele2013book}

É importante que os profissionais se apropriem dos usos dessas ferramentas e despertem para a reflexão de sua prática em cima das análises desses laços formado e de onde e como o paciente consegue apoio, a fim de reorientarem suas ações e compartilharem as alternativas de cuidado por parte de toda a equipe.