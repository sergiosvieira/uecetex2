\chapter{Metodologia}
\label{chap:metodologia}

\section{Desenho do Estudo}

Com abordagem qualitativa, é um estudo do tipo exploratório descritivo, pois preza pela explicação detalhada do assunto, dos fenômenos e dos elementos que o envolvem; definindo o campo de análise, os problemas e as questões iniciais, estabelecendo os contatos iniciais para entrada em campo, localizando os participantes e estabelecendo mais precisamente os procedimentos e instrumentos de coleta de dados \cite{andre2013estudo,augusto2013pesquisa}.

Escolheu-se evidenciar a abordagem na perspectiva qualitativa na qual atribui fundamental importância aos cenários naturais, os depoimentos e significados transmitidos pelos atores sociais envolvidos; sendo estes que constituem a fonte direta de dados, o pesquisador como principal instrumento e os dados coletados predominantemente descritivos \cite{augusto2013pesquisa,creswell2010projeto,vieira2005pesquisa}. À vista disso, a partir dos depoimentos dos sujeitos presentes no ambiente pesquisado, é possível uma interpretação mais realista da vivência deste campo; resultando em uma análise coerente do objeto de estudo e em uma possível busca de resolução de problemas observados. 

\section{Local e participantes do estudo}

Fizeram parte do estudo as seis enfermeiras do CPN de Maracanaú. Teve-se como critérios de inclusão para o presente estudo: ser enfermeiro atuante no CPN há mais de seis meses; prestar assistência à parturiente e puérpera; estar em boas condições físicas e psicológicas. Como critérios de exclusão teve-se: enfermeiro em férias ou licença no momento da coleta de informações, enfermeiros que não prestavam assistência direta às parturientes e que não trabalhavam no CPN. Salienta-se que todas as enfermeiras do serviço participaram do estudo, visto que não se enquadraram nos critérios de exclusão.

O CPN do Hospital da Mulher e da Criança, que faz parte do Complexo Hospitalar Dr. João Elísio de Holanda, é resultado de um convênio com o Ministério da Saúde, preconizando o parto humanizado. Conta com a banheira de parto normal humanizado na água e uma assistência com equipe multidisciplinar capacitada para garantir a saúde e bem-estar das mulheres.  Atende os municípios de Maracanaú, Maranguape, Pacatuba, Guaiúba, Acarape, Redenção, Barreira e Palmácia; e realiza mais de 300 partos por mês, apresentando índice inferior a 40\% de partos cirúrgicos, resultando em uma unidade destaque, na região nordeste, na Rede Cegonha. \cite{maracanau2019}.

A pesquisa considerou o CPN de Maracanaú, por ter sido o primeiro habilitado pelo Ministério da Saúde no Ceará. Localizado no interior do Estado, configura-se como importante ponto de atenção, prestando assistência a partos de risco habitual aos municípios que compõe sua respectiva região de saúde. 

\section{Período da pesquisa}

A pesquisa ocorreu em 2018, com a coleta dos dados entre os meses de agosto a setembro.

\section{Métodos e procedimentos}

Foi realizado contato com a diretoria da instituição para anuência da pesquisa. Posteriormente, foi feita a comunicação com profissionais diretamente na instituição e entregou-se o Termo de Consentimento Livre e Esclarecido (TCLE) (Apêndice A) para assinatura. Posteriormente, foi feita a realização da entrevista individual semiestruturada, em horário escolhido pelos participantes. 

A entrevista é um método utilizado na coleta de dados, principalmente nas pesquisas qualitativas, que se destaca como um importante instrumento para obter informações e opiniões através da fala individual dos sujeitos acerca da temática, revelando condições estruturais, sistemas de valores, normas e símbolos e transmite, por um porta-voz, representações de determinados grupos (AUGUSTO et al., 2014; MINAYO, 2014). 

Para o registro das informações construídas, utilizou-se um gravador para a entrevista (Apêndice B) realizada em sala disponibilizada para este fim no CPN. Este recurso de áudio foi utilizado mediante a autorização dos participantes.

\section{Análise e discussão dos resultados}

A análise temática é uma das maneiras que melhor se adequa às pesquisas qualitativas; em vista disso, será utilizada como norteadora na construção das discussões. A aplicação dessa técnica constitui-se em três etapas: pré-análise, exploração do material ou codificação e tratamento dos resultados obtidos/interpretação.

Na etapa da pré-análise, o pesquisador iniciará com a realização de uma atividade conhecida como ``leitura flutuante'', atividade esta que objetivou gerar impressões iniciais acerca do material a ser analisado, correlacionando o material encontrado com os pressupostos iniciais e o referencial teórico-político escolhido para esse estudo. 

Posteriormente, na etapa da exploração do material, as informações contidas no material serão codificadas, ou seja, recorta-se o texto em busca de elaborar categorias temáticas. E, por último, na fase do tratamento dos resultados e interpretação, serão analisados os dados obtidos e a partir dos temas relevantes, são elaboradas as categorias temáticas, entendidas como expressões ou palavras significativas que expressam o conteúdo da fala \cite{minayo1989desafio}.

\section{Aspectos éticos}

Em conformidade com a Resolução n. 466, de 12 de dezembro de 2012, do Conselho Nacional de Saúde \cite{brazil2011portaria}, que trata de ética em pesquisa envolvendo seres humanos, o estudo foi aprovado pelo Comitê de ética da Universidade Estadual do Ceará por meio da Plataforma Brasil, sob n. 2.195.430. Salienta-se que o \acrshort{TCLE} foi lido durante a entrevista e entregue para o participante pessoalmente.

Os riscos para os participantes da pesquisa foram mínimos, oriundos da discussão nas entrevistas individuais quando poderiam emergir lembranças de situações e fatos ligados às práticas cotidianas no trabalho. Porém os pesquisadores tentaram minorá-los ao explicar todas as etapas da pesquisa e as temáticas abordadas, preparando os participantes para as possíveis discussões que ocorreram. 

Os principais benefícios de participação na pesquisa estão relacionados às possibilidades de análise da percepção do enfermeiro acerca da violência obstétrica no CPN e a partir disso reflexão sobre as práticas com mudanças positivas no processo de trabalho destes. 

Os resultados serão devolvidos aos participantes/instituições por meio de apresentação em evento na Universidade Estadual do Ceará, que tratará do impacto dos CPN no Estado e divulgados mediante apresentação em eventos e publicações em periódicos.
