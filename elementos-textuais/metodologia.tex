\chapter{Metodologia}
\label{chap:metodologia}

Neste trabalho, realizou-se uma revisão integrativa de artigos relacionados às ações tomadas por enfermeiros nas consultas a gestante com sífilis gestacional de modo a aumentar a adesão ao tratamento.

As revisões são métodos de pesquisa que servem para fornecer conhecimentos sobre um problema de pesquisa específico, sendo esse conhecimento transformado em prática posteriormente. No que se refere a revisão integrativa, esta é usada para sintetizar informações obtidas a respeito de um assunto de maneira sistemática, ordenada e abrangente. Sendo abrangente ela pode ter diferentes fins: definir conceitos, rever teorias ou analisar métodos de estudos sobre um tema específico. \cite{mendes2005revisao}

Seguiram-se os seguintes passos nesta revisão: identificação do problema com delineamento claro do propósito do estudo, pesquisa nas bases de dados por literatura pertinente utilizando descritores, formulação da base de dados e uso dos critérios de inclusão e exclusão para seleção dos artigos e avaliação dos achados com posterior apresentação dos resultados. \cite[p. 04]{teixeira2014integrative}.

Os dados foram levantados por meio de pesquisa pareada e independente no site da \acrshort{BVS} (\acrlong{BVS}), em janeiro de 2019 utilizando-se os descritores presentes no \acrshort{DeCS} (\acrlong{DeCS}): sífilis, enfermagem e pré-natal, colocando-se o booleando AND entre os termos. Foram encontradas 30 publicações. Após a aplicação dos critérios de inclusão e exclusão, selecionaram-se os artigos que compuseram a amostra do estudo.

Optou-se como critérios de inclusão: estudos em língua portuguesa, inglesa e espanhola; publicados de 2008 a 2019; realizados no Brasil, disponíveis online e na íntegra que abordavam as ações tomadas por enfermeiros no pré-natal a gestantes com sífilis para a adesão de tratamento de sífilis gestacional. 

Os critérios de exclusão adotados foram: publicações que não fossem artigos científicos, outras revisões ou reflexões, estudos que não foram desenvolvidos no Brasil, estudos repetidos ou que não respondessem o objetivo desta pesquisa. 

Das 30 publicações iniciais, 21 estavam disponibilizadas integralmente. Entretanto,  apenas 14 eram artigos. As outras publicações eram 06 teses e 01 recurso de educação aberto. Foram analisados os 14 artigos, dos quais 06 foram escolhidos com base nos critérios para compor a amostra do estudo. Os excluídos tratavam-se de 02 revisões, 01 artigo de reflexão, 01 artigo repetido, 01 não disponível online, 01 anais de congresso e 01 estudo não realizado no Brasil. 

Quanto às bases de dados, os artigos que compuseram a amostra estavam disponíveis nas bases de dados \acrlong{BDENF} (\acrshort{BDENF}) - Enfermagem (5) e \acrlong{LILACS} (\acrshort{LILACS}) (1).

Depois da leitura e fichamento dos trabalhos pesquisados, realizou-se à análise descritiva e de conteúdo dos mesmos, contribuindo para a reflexão sobre o tema abordado neste trabalho. 

% Para atingir os objetivos desta pesquisa, foi escolhido o método de revisão integrativa que ``combina dados da literatura teórica e empírica, além de incorporar um vasto leque de propósitos, como: definição de conceitos, revisão de teorias e evidências, análise de problemas metodológicos.'' \cite[p. 04]{teixeira2014integrative}.

% Em meio a uma variedade de formas de se realizar uma revisão integrativa, alguns aspectos devem ser respeitados de forma a garantir um trabalho com rigor metodológico. Assim, essa revisão será constituída dos seguintes passos: identificação do problema com claro delineamento do propósito do estudo, busca nas bases de dados por literatura pertinente utilizando descritores ou palavras-chave, formulação da base de dados e uso dos critérios de inclusão e exclusão para seleção dos artigos e avaliação dos achados com posterior apresentação dos resultados. \cite[p. 04]{teixeira2014integrative}.

% Os dados serão levantados nas bases de dados \acrlong{MEDLINE} (\acrshort{MEDLINE}), \acrlong{LILACS} (\acrshort{LILACS}) e \acrlong{SciELO} (\acrshort{SciELO}), consideradas as principais da área da saúde brasileira. 

% Visando garantir o rigor metodológico, a busca dos artigos será feita pelos dois revisores concomitantemente e serão utilizados os termos considerados descritores no \acrlong{DeCS} (\acrshort{DeCS}): Rede Social, Apoio Social, Cuidado de Enfermagem e Enfermagem da Família. 

% Como critérios de inclusão optou-se por: estudos em língua portuguesa, inglesa e espanhola; publicados de 2005 a 2015; que abordem os conceitos de redes sociais e que tragam o enfermeiro envolvido nesse processo de formação; que sejam de livre acesso, disponíveis online e na íntegra, e que foram produzidos no Brasil, pois é a realidade que se quer abordar. 

% Como critérios de exclusão tem-se: estudos desenvolvidos em outros países, guidelines, artigos de revisão, protocolos, estudos que aparecerem em mais de uma base de dados, que estejam fora do recorte de tempo e que não respondam aos objetivos da pesquisa. 

% Após a coleta do material ambos os revisores, separadamente, darão início a suas leituras e executarão a extração dos dados que respondam aos objetivos. Os achados serão confrontados, verificando-se onde os estudos convergem e divergem, e posteriormente, eles serão organizados analisados e categorizados tematicamente de acordo com os temas que mais surgirem.  
