\chapter{Metodologia}
\label{chap:metodologia}

\section{Tipo de Estudo}
Trata-se de um estudo de caso de abordagem qualitativa e exploratória. A pesquisa exploratória segundo \cite{leopardi2001metodologia} permite ao investigador aumentar sua experiência em torno de um determinado problema. Consiste em explorar tipicamente a primeira aproximação de um tema e visa criar maior familiaridade em relação a um fato ou fenômeno. 

Já um estudo de caso, para \cite{leopardi2001metodologia}, trata-se de uma investigação sobre um único evento ou situação (caso) em que se busca aprofundamento dos dados  sem preocupação com a frequência de sua ocorrência. Quanto a pesquisa qualitativa, \cite{leopardi2001metodologia} enfoca a preocupação com a informação que surge a partir de pessoas que estão diretamente envolvidas com a experiência estudada. Considerando-se a rede social que se forma a partir de um indivíduo e compreensão da realidade em que esta é formada, o tipo de estudo adequa-se a situação que se deseja estudar.

\section{Cenário de Estudo}
A coleta de dados realizou-se em Icapuí-Ce, na Unidade Básica de Saúde da localidade de Barreira, local onde a enfermeira escolhida para análise da rede desenvolvia suas atividades. Além deste local, foram utilizados também os locais de trabalho dos outros trabalhadores da saúde citados por ela ou suas residências de acordo com a conveniência dos entrevistados. Dentre estes locais de trabalho estão a Secretaria de Saúde do município, o Centro de Atenção Psicossocial e o Hospital Municipal Maria Idalina Rodrigues de Medeiros. 

O município de Icapuí está situado no extremo Leste do estado do Ceará, a 210,05 km da capital, Fortaleza \cite{der}. Tem como fronteira a norte com o Oceano Atlântico, a leste com o estado do Rio Grande do Norte, e no Ceará com a cidade de Aracati. É subdividada em três (3) distritos: Icapuí (sede), Ibicuitaba e Manibu. Possui uma população de 18.392 habitantes \cite{censo2010disponivel}.
  
O município conta com 8 equipes de \acrshort{ESF}, 8 \acrshort{UBS}, um \acrlong{NASF} (\acrshort{NASF}) da Residência Multiprofissional, um \acrlong{CAPS} (\acrshort{CAPS}) Geral e um Hospital Municipal. Sendo um município com atividade pesqueira e de beneficiamento de pescados como uma das principais fontes de renda, apresenta comunidades em localidades mais afastadas de seu centro, fazendo-se assim necessária uma boa articulação dos trabalhadores da saúde entre si para resolução das demandas diárias. Além disso, a rede conta com as pactuações junto ao município de Aracati.  

\section{Participantes da Pesquisa}
Para o estudo foi-se escolhida por conveniência a enfermeira da \acrlong{UBS} de Barreira e a partir dela, cinco (5) atores que a mesma considerou serem os mais representativos para sua rede social no cuidado a hipertensos e diabéticos. Estes atores, por sua vez, também nomearam cinco (5) atores cada um. Ficando a rede nesse nível, delineada para facilitar a análise qualitativa dos dados. 

Ressalta-se que todos os participantes apontados como atores foram trabalhadores da saúde com diferentes vínculos com o município ou com sua rede de assistência. 

\section{Período e Instrumento de Coleta de Dados}
A coleta realizou-se nos dias 29 e 30 de junho de 2015. Foi elaborada uma entrevista semi-estruturada para permitir aos participantes a exposição livre a respeito do tema, bem como a investigação mais ampla por parte do pesquisador. Após serem dadas as informações sobre a pesquisa e sobre a coleta de dados, foi apresentado aos participantes o \acrlong{TCLE} (\acrshort{TCLE}). Após a assinatura do \acrshort{TCLE}  e com a autorização dos participantes, as entrevistas foram gravadas em áudio e transcritas na íntegra para posterior análise. 

O instrumento elaborado passou por uma testagem prévia de seu roteiro com profissionais não participantes da pesquisa a fim de detectar situações que fugissem ao tema de interesse ou que influenciassem o discurso do participante, criando enviesamentos. Nele constavam questionamentos a respeito da formação dos participantes, vínculo empregatício com o município, tempo de atividade no Sistema de Saúde municipal, tipos de atendimentos prestados aos pacientes hipertensos e diabéticos, como os profissionais percebiam o sistema de referência e contra-referência municipal a esses pacientes, situações em que tiveram que ativar outros atores para a continuidade do cuidado a esses pacientes e as cinco pessoas que elas julgassem serem as mais ativadas por elas para suas redes sociais no que tange ao desempenho de duas ações de cuidado.  

\section{Organização e Análise de Dados}
As entrevistas foram transcritas na íntegra e organizadas em arquivos no Word. Foi utilizada a técnica de \acrlong{ARS} (\acrshort{ARS}) com o auxílio do software UCINET versão 6.18 e Netdraw \cite{ucinet}. Os atores sociais citados tiveram os nomes decodificados para a análise dos dados no software. O grafo gerado foi analisado visando identificar quais foram os profissionais mais acessados pela enfermeira cuja rede foi inicialmente analisada, como também a localização da mesma profissional dentro da rede que se construiu a partir dos atores posteriormente citados.

Quanto às medidas utilizadas para a análise, estas surgiram a partir dos objetivos a que se propôs o estudo: identificar os atores e localizar a posição da enfermeira origem na rede formada. O grafo originado permitiu ainda o uso das medidas: densidade, 
grau de centralidade, grau de proximidade e grau de intermediação.

Além disso, foi feita a leitura do transcrito das entrevistas de onde emergiram os temas que foram discutidos com base na literatura pertinente. 

Na fase de leitura, \cite{campos2004metodo} coloca que esta a princípio, é feita sem compromisso objetivo de  sistematização, mas sim se tentando apreender de uma forma global as ideias principais e os seus significados gerais.  

Assim, elas podem ser previamente pensadas com base nos objetivos da pesquisa ou podem  surgir das respostas dos sujeitos, fornecendo assim uma ampliação do problema estudado, revelando através do discurso dos participantes pontos não pensados pelo pesquisador. No estudo em curso as respostas dos sujeitos foram fomentadoras importantes.

\section{Aspectos Éticos e Legais}
A pesquisa teve início com a aprovação do projeto Redes sociais no trabalho de enfermeiros da Atenção Básica: um estudo em municípios do Rio de Janeiro e Ceará pelo \acrlong{CEP} (\acrshort{CEP}) da Universidade Estadual do Ceará, instituição coparticipante do estudo, com Parecer número 818.029/2014, CAAE 33423114.9.3001.5534 (ANEXO B) e autorização das \acrlong{UAPS} (\acrshort{UAPS}) através da Anuência falta a anuência (ANEXO A) concedida pela \acrlong{SMS} (\acrshort{SMS}) de Icapuí. Foram respeitados os aspectos éticos e legais preconizados pela Resolução 466/2012 do \acrlong{CNS} (\acrshort{CNS}). 

Mediante a assinatura no \acrshort{TCLE} e a permissão dos participantes da pesquisa, as entrevistas foram gravadas em áudio e posteriormente transcritas na íntegra. Garantiu-se o sigilo das informações fornecidas, bem como o esclarecimento de quaisquer dúvidas a respeito do estudo, do roteiro da entrevista e dos benefícios, riscos e o direito dos participantes em desistirem de seu consentimento em qualquer momento da pesquisa. Para manter a identidade dos participantes sob sigilo, seus nomes foram substituídos por siglas nas transcrições e nos resultados encontrados.
