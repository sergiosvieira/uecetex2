\chapter{Metodologia}
\label{chap:metodologia}

Para atingir os objetivos desta pesquisa, foi escolhido o método de revisão integrativa que ``combina dados da literatura teórica e empírica, além de incorporar um vasto leque de propósitos, como: definição de conceitos, revisão de teorias e evidências, análise de problemas metodológicos.'' \cite[p. 04]{teixeira2014integrative}.

Em meio a uma variedade de formas de se realizar uma revisão integrativa, alguns aspectos devem ser respeitados de forma a garantir um trabalho com rigor metodológico. Assim, essa revisão será constituída dos seguintes passos: identificação do problema com claro delineamento do propósito do estudo, busca nas bases de dados por literatura pertinente utilizando descritores ou palavras-chave, formulação da base de dados e uso dos critérios de inclusão e exclusão para seleção dos artigos e avaliação dos achados com posterior apresentação dos resultados. \cite[p. 04]{teixeira2014integrative}.

Os dados serão levantados nas bases de dados \acrlong{MEDLINE} (\acrshort{MEDLINE}), \acrlong{LILACS} (\acrshort{LILACS}) e \acrlong{SciELO} (\acrshort{SciELO}), consideradas as principais da área da saúde brasileira. 

Visando garantir o rigor metodológico, a busca dos artigos será feita pelos dois revisores concomitantemente e serão utilizados os termos considerados descritores no \acrlong{DeCS} (\acrshort{DeCS}): Rede Social, Apoio Social, Cuidado de Enfermagem e Enfermagem da Família. 

Como critérios de inclusão optou-se por: estudos em língua portuguesa, inglesa e espanhola; publicados de 2005 a 2015; que abordem os conceitos de redes sociais e que tragam o enfermeiro envolvido nesse processo de formação; que sejam de livre acesso, disponíveis online e na íntegra, e que foram produzidos no Brasil, pois é a realidade que se quer abordar. 

Como critérios de exclusão tem-se: estudos desenvolvidos em outros países, guidelines, artigos de revisão, protocolos, estudos que aparecerem em mais de uma base de dados, que estejam fora do recorte de tempo e que não respondam aos objetivos da pesquisa. 

Após a coleta do material ambos os revisores, separadamente, darão início a suas leituras e executarão a extração dos dados que respondam aos objetivos. Os achados serão confrontados, verificando-se onde os estudos convergem e divergem, e posteriormente, eles serão organizados analisados e categorizados tematicamente de acordo com os temas que mais surgirem.  
