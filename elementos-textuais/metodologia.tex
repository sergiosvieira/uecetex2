\chapter{Metodologia}
\label{chap:metodologia}

Para atingir os objetivos desta pesquisa, foi escolhido o método de revisão integrativa que ``combina dados da literatura teórica e empírica, além de incorporar um vasto leque de propósitos, como: definição de conceitos, revisão de teorias e evidências, análise de problemas metodológicos.'' \cite[p. 04]{teixeira2014integrative}.

Em meio a uma variedade de formas de se realizar uma revisão integrativa, alguns aspectos devem ser respeitados de forma a garantir um trabalho com rigor metodológico. Assim, essa revisão será constituída dos seguintes passos: identificação do problema com claro delineamento do propósito do estudo, busca nas bases de dados por literatura pertinente utilizando descritores ou palavras-chave, formulação da base de dados e uso dos critérios de inclusão e exclusão para seleção dos artigos e avaliação dos achados com posterior apresentação dos resultados. \cite[p. 04]{teixeira2014integrative}.

Os dados serão levantados nas bases de dados \acrlong{MEDLINE} (\acrshort{MEDLINE}), \acrlong{LILACS} (\acrshort{LILACS}) e \acrlong{SciELO} (\acrshort{SciELO}), consideradas as principais da área da saúde brasileira. 

Visando garantir o rigor metodológico, a busca dos artigos será feita pelos dois revisores concomitantemente e serão utilizados os termos considerados descritores no \acrlong{DeCS} (\acrshort{DeCS}): Rede Social, Apoio Social, Cuidado de Enfermagem e Enfermagem da Família. 

Como critérios de inclusão optou-se por: estudos em língua portuguesa, inglesa e espanhola; publicados de 2005 a 2015; que abordem os conceitos de redes sociais e que tragam o enfermeiro envolvido nesse processo de formação; que sejam de livre acesso, disponíveis online e na íntegra, e que foram produzidos no Brasil, pois é a realidade que se quer abordar. 

Como critérios de exclusão tem-se: estudos desenvolvidos em outros países, guidelines, artigos de revisão, protocolos, estudos que aparecerem em mais de uma base de dados, que estejam fora do recorte de tempo e que não respondam aos objetivos da pesquisa. 

Após a coleta do material ambos os revisores, separadamente, darão início a suas leituras e executarão a extração dos dados que respondam aos objetivos. Os achados serão confrontados, verificando-se onde os estudos convergem e divergem, e posteriormente, eles serão organizados analisados e categorizados tematicamente de acordo com os temas que mais surgirem.  


\begin{table}[h!]
	\Caption{\label{tabela-ibge} Um Exemplo de tabela alinhada que pode ser longa ou curta, conforme padrão IBGE. conforme padrão IBGE. conforme padrão IBGE. conforme padrão IBGE. conforme padrão IBGE. conforme padrão IBGE. conforme padrão IBGE. conforme padrão IBGE. conforme padrão IBGE. conforme padrão IBGE. conforme padrão IBGE.}%
	\IBGEtab{}{%
		\begin{tabular}{ccc}
			\toprule
			Nome & Nascimento & Documento \\
			\midrule \midrule
			Maria da Silva & 11/11/1111 & 111.111.111-11 \\
			Maria da Silva & 11/11/1111 & 111.111.111-11 \\
			Maria da Silva & 11/11/1111 & 111.111.111-11 \\
			\bottomrule
		\end{tabular}%
	}{%
	\Fonte{Produzido pelos autores}%
	\Nota{Esta éuma nota, que diz que os dados são baseados na
		regressão linear.}%
	\Nota[Anotações]{Uma anotação adicional, seguida de várias outras.}%
}
\end{table}

\cite{Huetal2000} \lipsum[2] 

\section{Exemplo de Algoritmos e Figuras}
\label{sec:exemplo-de-algoritmos-e-figuras}

\lipsum[2]

\begin{algorithm}[h!]
	\Entrada{o proprio texto}
	\Saida{como escrever algoritmos com \LaTeX2e }
	\Inicio{
		inicializa\c{c}\~ao\;
		\Repita{fim do texto}{
			leia o atual\;
			\Se{entendeu}{
				vá para o próximo\;
				próximo se torna o atual\;}
			\Senao{volte ao início da seção\;}
		}
	}
	\caption{Como escrever algoritmos no \LaTeX2e}
\end{algorithm}

\lipsum[2]
%\begin{algorithm}[H]
%	\Entrada{o proprio texto}
%	\Saida{como escrever algoritmos com \LaTeX2e }
%	\Inicio{
%		inicializa\c{c}\~ao\;
%		\Repita{fim do texto}{
%			leia o atual\;
%			\Se{entendeu}{
%				vá para o próximo\;
%				próximo se torna o atual\;}
%			\Senao{volte ao início da seção\;}
%		}
%	}
%	\caption{Exemplo de Algoritmo Versao 02}
%\end{algorithm}

%\begin{algorithm}
%	\begin{algorithmic}
%	\Entrada{o proprio texto}
%	\Saida{como escrever algoritmos com \LaTeX2e }	
%	\end{algorithmic}
%\end{algorithm}

Exemplo de alíneas com números:

\begin{alineascomnumero}
	\item Lorem ipsum dolor sit amet, consectetur adipiscing elit. Nunc dictum sed tortor nec viverra.
	\item Praesent vitae nulla varius, pulvinar quam at, dapibus nisi. Aenean in commodo tellus. Mauris molestie est sed justo malesuada, quis feugiat tellus venenatis.
	\item Praesent quis erat eleifend, lacinia turpis in, tristique tellus. Nunc dictum sed tortor nec viverra.
	\item Mauris facilisis odio eu ornare tempor. Nunc dictum sed tortor nec viverra.
	\item Curabitur convallis odio at eros consequat pretium.
\end{alineascomnumero}

\lipsum[12]

\begin{table}[h!]	
	\centering
	\Caption{\label{tab:internal}Internal exon scores}	
	\IBGEtab{}{
		\begin{tabular}{cll}
			\toprule
			Ranking & Exon Coverage & Splice Site Support\\
			\midrule \midrule
			E1 & Complete coverage by a single transcript & Both splice sites\\
			E2 & Complete coverage by more than a single transcript & Both splice sites\\
			E3 & Partial coverage & Both splice sites\\
			E4 & Partial coverage & One splice site\\
			E5 & Complete or partial coverage & No splice sites\\
			E6 & No coverage & No splice sites\\
			\bottomrule
		\end{tabular}
	}{
	\Fonte{os autores}
}
\end{table}

\lipsum[2] Referenciando a \autoref{tab:internal} \lipsum[2]

\index{figuras}Figuras podem ser criadas diretamente em LaTeX,
como o exemplo da \ref{fig-grafico-1}.

\begin{figure}[h!]
	\centering
	\Caption{\label{fig-grafico-1}Produção anual das dissertações de mestrado e teses de doutorado entre os anos de 1990 e 2008}		
	\IBGEtab{}{
		\fbox{\includegraphics[scale=0.5]{figuras/figura-3}}
	}{
	\Fonte{os autores}
}
\end{figure}

Ou então figuras podem ser incorporadas de arquivos externos, como é o caso da \autoref{fig-grafico-1}. Se a figura que ser incluída se tratar de um diagrama, um gráfico ou uma ilustração que você mesmo produza, priorize o uso de imagens vetoriais no formato PDF. Com isso, o tamanho do arquivo final do trabalho será menor, e as imagens terão uma apresentação melhor, principalmente quando impressas, uma vez que imagens vetorias são perfeitamente escaláveis para qualquer dimensão. Nesse caso, se for utilizar o Microsoft Excel para produzir gráficos, ou o Microsoft Word para produzir ilustrações, exporte-os como PDF e os incorpore ao documento conforme o exemplo abaixo. No entanto, para manter a coerência no uso de software livre (já que você está usando LaTeX e abnTeX),  teste a ferramenta InkScape\index{InkScape}. ao CorelDraw\index{CorelDraw} ou ao Adobe Illustrator\index{Adobe! Illustrator}.  De todo modo, caso não seja possível  utilizar arquivos de imagens como PDF, utilize qualquer outro formato, como JPEG, GIF, BMP, etc.  Nesse caso, você pode tentar aprimorar as imagens incorporadas com o software livre \index{Gimp}Gimp. Ele é uma alternativa livre ao Adobe Photoshop\index{Adobe! Photoshop}.

\section{Usando Fórmulas Matemáticas}

\lipsum[2]

	\begin{equation}
		\begin{aligned}
			x = a_0 + \cfrac{1}{a_1
				+ \cfrac{1}{a_2
					+ \cfrac{1}{a_3 + \cfrac{1}{a_4} } } }
		\end{aligned}
	\end{equation}

\lipsum[3]

	\begin{equation}
		\begin{aligned}
			k_{n+1} = n^2 + k_n^2 - k_{n-1}
		\end{aligned}
	\end{equation}

\lipsum[4]

	\begin{equation}
		\begin{aligned}
			\cos (2\theta) = \cos^2 \theta - \sin^2 \theta
		\end{aligned}
	\end{equation}
	
\lipsum[5]

	\begin{equation}
		\begin{aligned}
			A_{m,n} =
			\begin{pmatrix}
			a_{1,1} & a_{1,2} & \cdots & a_{1,n} \\
			a_{2,1} & a_{2,2} & \cdots & a_{2,n} \\
			\vdots  & \vdots  & \ddots & \vdots  \\
			a_{m,1} & a_{m,2} & \cdots & a_{m,n}
			\end{pmatrix}
		\end{aligned}
	\end{equation}

\lipsum[6]

	\begin{equation}
		\begin{aligned}
			f(n) = \left\{ 
			\begin{array}{l l}
			n/2 & \quad \text{if $n$ is even}\\
			-(n+1)/2 & \quad \text{if $n$ is odd}
			\end{array} \right.
		\end{aligned}
	\end{equation}
	
\lipsum[7]

\section{Usando Algoritmos}

\lipsum[8]

\begin{algorithm}[h!]
	\Caption{\label{alg:algoritmo_de_colonica_de_formigas}Algoritmo de Otimização por Colônia de Formiga}
	\Inicio{
		Atribua os valores dos parâmetros\;
		Inicialize as trilhas de feromônios\;
		\Enqto{não atingir o critério de parada}{
			\Para{cada formiga}{
				Construa as Soluções\;
			}
			Aplique Busca Local (Opcional)\;
			Atualize o Feromônio\;
		}	
	}		
\end{algorithm}

\lipsum[9]

\section{Usando Código-fonte}

\lipsum[10]

\lstinputlisting[language=C++,caption={Hello World em C++}]{figuras/main.cpp}

\lipsum[11]

\begin{lstlisting}[language=Java,caption={Hello World em Java}]
public class HelloWorld {
	public static void main(String[] args) {
		System.out.println("Hello World!");
	}
}
\end{lstlisting}

\lipsum[11]