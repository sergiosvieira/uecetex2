\chapter{Análise e Discussão dos Resultados}
\label{chap:resultados}

Apresentam-se algumas informações que buscar caracterizar as participantes. As profissionais entrevistadas tinham idade entre 26 a 59 anos, todas do sexo feminino, com tempo de trabalho no CPN de Maracanaú variando de 6 meses a 10 anos, com média de 6,5 anos. Verificou-se que todas as enfermeiras possuíam especialização em Enfermagem Obstétrica e cursos em áreas afins. Portanto, todas estavam legalmente aptas a acompanhar partos de risco habitual sem a exigência de presença de outros profissionais.

Resultados semelhantes foram encontrados em outro estudo com 20 profissionais da saúde, realizado em uma maternidade de referência no município de Caxias, situado ao leste do estado do Maranhão, Nordeste do Brasil. \cite{cardoso2017violencia}.

\section{Percepção da enfermeira obstetra acerca da violência obstétrica}

Por meio da análise dos discursos das enfermeiras obstetras, observou-se que nos discursos de quatro enfermeiras possuem uma percepção sobre a violência obstétrica e nelas remetem muito o que tem no manual de boas práticas da atenção ao parto e ao nascimento da OMS sobre a categoria D que são as práticas frequentemente usadas de modo inadequado. \cite{world1996world}.

\begin{citacao}
``Violência obstétrica pra mim é fazer um parto, tentar agilizar um parto sem a necessidade; não deixar a natureza agir; não dar aquelas orientações necessárias das posições, deixar de ofertar para a paciente as posições mais adequadas, mostrando pra ela que vai amenizar aquela dor, e que o bebê vai nascer tranquilo. Fazer um Kristeller pra mim é uma violência obstétrica, pra mim, fazer uma episio também é uma violência obstétrica. Existem momentos, raridades, que necessite. [...].'' (E-1)
\end{citacao}

\begin{citacao}
``Eu acho que é uma coisa que a paciente não aceita (pausa). Violência seria assim, deixa eu ver ... acho que tratar mal, é fazer uma, porque assim, já convencionou de achar que fazer Kristeller é uma violência, mas eu acho que a violência é até você falar grosseiro com a paciente, tudo é violência, eu acho. [...].'' (E-3)
\end{citacao}

\begin{citacao}
``Eu acho que é quando se falta com respeito à mulher. Para mim, humanização, cuidado, a primeira palavra que me vem à cabeça é respeito. Pra mim, violência obstétrica é quando se falta com respeito, é quando você faz algo que está violando o corpo daquela mulher; quando você faz algo sem a autorização dela; quando você faz algo desnecessário; quando você fala algo que a  falte com respeito ou que lhe possa ofender, entendeu? Então, pra mim, isso já entra na violência obstétrica.'' (E-4)
\end{citacao}

Entende-se por violência obstétrica ações desumanas praticadas por profissionais de saúde apropriando-se do corpo e dos processos reprodutivos das mulheres, impactando negativamente na qualidade de vida das mesmas, resultando em perda de autonomia e capacidade de decidir livremente sobre seu corpo e sexualidade. Como por exemplo, o uso abusivo de medicamentos, intervenções desnecessárias e tornando o processo natural do parto em doença. \cite[p. 3]{diniz2015abuse}.

De acordo com SAUAUA e SERRA (\citeyear{sauaia2016dor}), a violência obstétrica é uma forma de violência contra mulher e implica em violação de direitos humanos, caracterizada pela imposição de ações danosas à integridade física e mental das gestantes, perpetrada pelos profissionais de saúde, bem como pelas instituições (públicas e privadas) nas quais tais mulheres são atendidas. 

Estudo realizado por \citeauthor{leal2018percepccao} (\citeyear{leal2018percepccao}), vem ratificar o que está sendo observado nas falas das enfermeiras do CPN de Maracanaú sobre as suas percepções a respeito da violência obstétrica. Os relatos a seguir reproduzem quase o mesmo conhecimento sobre o assunto em questão: 

\begin{citacao}
``Os procedimentos e as atitudes que caracterizam violência obstétrica podem ser [...] manobra de Kristeller, episiotomia sem consentimento, toques doloridos e sucessivos por vários avaliadores e uso indiscriminado de soro com ocitocina.'' (E14) 
\end{citacao}

\begin{citacao}
``A violência psicológica, quando utilizamos palavras inapropriadas para constranger a mulher, também é uma violência obstétrica.'' (E05) 
\end{citacao}

\begin{citacao}
``Algumas vezes, o profissional pressiona a parturiente durante o trabalho de parto, afirmando que o bebê nascera com alguma sequela por culpa dela.'' (E08)
\end{citacao}

Diante do que foi exposto nessa categoria, venho por meio das falas das enfermeiras do CPN de Maracanaú comprovar as suas percepções a respeito da violência obstétrica. Relatos estes que evidenciam indicadores epidemiológicos com experiências exitosas compreendeu que a humanização do parto e do nascimento tem a perspectiva de um novo modelo que deverá ser a prioridade em suas ações. \cite[p.19]{vico2017avaliaccao}.

\section{Prática de violência no cotidiano do CPN}

Nos discursos de quatro enfermeiras, afirmam que não praticavam violência obstétrica, no entanto duas reconheceram que realizaram violência obstétrica contra a parturiente. 

\begin{citacao}
``Não. Eu procuro me vigiar até na forma de falar. Porque, a gente quer a mulher tá ali, num momento, uma bomba de hormônios, aí, o acompanhante, às vezes, não é aquela tranquilidade que deveria ser, e às vezes, fica aquela coisa, às vezes, as formas como você fala pra eles, você pode achar, ``ela foi grosseira, ela foi rude''.[...]'' (E-4)
\end{citacao}

\begin{citacao}
``Não, não. Há muito tempo eu já fazia o contato pele a pele, devido a ocitocina natural, para ajudar no delivramento da placenta. Mas, já fiz episio por achar que estava fazendo o bem, para poder não lacerar, mas hoje vejo que não há necessidade.'' (E-6)
\end{citacao}

\citeauthor{leal2018percepccao} (\citeyear{leal2018percepccao}), mostra em sua pesquisa, que algumas enfermeiras obstetras não reconhecem suas ações como uma prática violenta. Além disso, quando há o reconhecimento de tais procedimentos como uma prática nociva, existe a justificativa da ajuda à gestante para a realização das condutas, como mostram os relatos a seguir: 

\begin{citacao}
``Eu não vejo os procedimentos de rotina como uma violência obstétrica. O profissional que está conduzindo o parto é quem vai avaliar e decidir se precisa ou não intervir.'' (E04)
\end{citacao}

\begin{citacao}
``$[...]$ sei que as evidências científicas mostram que é melhor não fazer episiotomia, que a mulher pode ter dificuldade na cicatrização, alterar a sensibilidade da região e outros fatores. Mas dependendo do número de gestações, do tamanho do bebê, do tempo de trabalho de parto, acredito que é necessário fazer para resolver.'' (E12)
\end{citacao}

Em um estudo realizado por \citeauthor{cardoso2017violencia} (\citeyear{cardoso2017violencia}), apenas 15\% dos profissionais relataram já ter praticado ou que ainda praticam algum tipo de violência obstétrica. Os profissionais afirmam ter cometido esse tipo de ato violento, pois queriam intervir precipitadamente pelo medo de complicações indicando assim cesáreas, algumas vezes desnecessárias. 

Nota-se no discurso do profissional que admite já ter cometido violência obstétrica, porém sugere que a culpa de tal problema é devido ao sistema de saúde e/ou da cultura das gestantes/parturientes. \cite{cardoso2017violencia}. No estudo vigente as enfermeiras que responderam ter cometido violência obstétrica soaram mais verdadeiro, pois somos seres humanos passíveis a erros e no momento de estresse emocional onde estamos lidando com a vida podemos falar ou fazer algo que não condiz com as boas práticas de uma forma não humanizada.

\begin{citacao}
``Já, aconteceu uma vez, eu achei que foi uma violência, depois eu cai na real e percebi. Eu falei ``você vai matar seu filho'', eu disse isso sem querer, disse mesmo, porque ela fechando as pernas e eu feito louca desesperada, e a mulher, ``você ta entendendo''? [...].'' (E-3)
\end{citacao}

\begin{citacao}
``A gente sempre faz, porque assim, na hora do sufoco que você está, se o bebê demora a sair, a gente pode agir de alguma forma diferente, que na hora a gente nem sabe que a gente está praticando essa violência, e que a paciente pode achar que sim, ou a acompanhante pode achar e depois vir reclamar em relação ao nosso procedimento. [...].'' (E-5)
\end{citacao}

Todas as profissionais já presenciaram violência obstétrica no cotidiano do CPN, praticada pela categoria médica, como exemplo, Kristeller, episiotomia e redução do colo. Da mesma forma as possíveis ações que tem risco de transformar-se em violência obstétrica, como exemplo, o toque vaginal repetido foi o mais comentado nas falas das enfermeiras.

\begin{citacao}
``A questão de redução de colo, reduzir o colo; e Kristeller com episiotomia.'' (E-1)
\end{citacao}

\begin{citacao}
``Às vezes, um toque; às vezes, como eu falo, a redução do colo, a gente querendo reduzir o colo assim, brutalmente, várias vezes.'' (E-1)
\end{citacao}

\begin{citacao}
``Já vi Kristeller. Já vi episiotomia, que não necessitava daquilo. Eu acho que os que eu já vi foram esses. Mas assim, não foram de profissionais da Enfermagem. Os que eu presenciei tanto como profissional como na minha época de residente, não foram do pessoal da Enfermagem, foram esses dois, tanto Kristeller quanto episiotomia, sem necessidade e sem autorização da paciente.'' (E-4)
\end{citacao}

\begin{citacao}
``Acho que o toque exagerado, a questão de redução de colo, né? Questão, por exemplo, de às vezes colocar ocitocina sem real indicação, pra poder acelerar o trabalho de parto. [...].'' (E-4)
\end{citacao}

\begin{citacao}
``O Kristeller da vida, profissionais chegar e ficar fazendo toque sem nenhum preparo. Eu vejo muito as enfermeiras pedindo ``com licença'' para fazer o toque, porém vejo muitos médicos que fazem a redução de colo, rompeu a bolsa e depois foi se apresentar, e ainda falou assim: ``seu menino agora vai nascer, porque eu cheguei e vou fazer seu parto''. Vejo muitos ainda fazendo episio sem necessidade, apenas pra apressar o parto. Aí, a mulher no final ainda agradece, pois pensa que ela fez pelo bem dela. Não consigo entender essas coisas.'' (E-6)
\end{citacao}

Segundo \citeauthor{zanardo2017violencia} (\citeyear{zanardo2017violencia}), a aplicação de pressão na parte superior do útero que é conhecida como manobra de Kristeller teve uma ocorrência de 37\% e o corte na região do períneo conhecido como episiotomia ocorreu em 56\% dos partos. Esse número de intervenções foi considerado excessivo e não encontra respaldo científico em estudos internacionais. Além disso, muitas dessas práticas são associadas a risco de complicações, são dolorosas e desnecessárias. 

\citeauthor{cardoso2017violencia} (\citeyear{cardoso2017violencia}), observou em sua pesquisa com profissionais da saúde quando eram questionados em relação à percepção que tinham acerca do colega de trabalho, 80\% dos entrevistados referiram já ter presenciado colegas cometendo algum tipo de violência obstétrica, como evidencia as falas abaixo: 

\begin{citacao}
``Sim, já presencie todos os tipos de violência, acredito que muitos realizam por simplesmente arrogância e ignorância do profissional.'' (P12)
\end{citacao}

\begin{citacao}
``Sim, por colega médico, presenciei foi muito injusto pelo ato das maus palavras e mesmo mau atendimento de não dar uma explicação satisfatória à paciente.'' (P10) 
\end{citacao}

\begin{citacao}
``Sim, já presencie praticada por outros colegas e achei desumano.'' (P8)
\end{citacao}

Os profissionais desse estudo possuem a percepção de que os colegas de trabalho cometem violência obstétrica; mas os mesmos não reconhecem ter praticado atos de violência com as parturientes.

Saliento, a importância na capacitação dos profissionais da área da saúde, as mudanças no modelo assistencial da obstetrícia e na grade curricular durante a formação dos mesmos para que contemple uma visão social e humanística, de modo a ofertar as gestantes um processo de atenção adequado no ciclo gravídico-puerperal. Assim, promovemos as devidas mudanças nesses profissionais. \cite{sauaia2016dor,cardoso2017violencia}.

\section{Possibilidades de enfrentamento da violência obstétrica no CPN}

Em seu livro, Humanização do parto, \citeauthor{maia2010humanizaccao}, (\citeyear{maia2010humanizaccao}), fala sobre o \textit{ethos} profissional. Com isso, podemos compreender um pouco sobre a ``disputa de poder'' entre os profissionais médicos e enfermeiros. A autora relata que a medicina monopoliza um mercado com demanda ilimitada e legalmente protegido; a atividade desse profissional é individual e de responsabilidade intransferível, sendo o trabalho coletivo pouco desenvolvido; e a autonomia é crucial para o seu cotidiano de trabalho. 

No caso da obstetrícia, o corpo de conhecimentos não é exclusivo da medicina. Antes da constituição da especialidade em obstétrica para médicos, as parteiras já exerciam a ``arte de partejar'' e, atualmente, as enfermeiras obstetras e as obstetrizes também reivindicam a sua expertise sobre o parto normal sem complicações. \cite[p.69]{maia2010humanizaccao}.

\begin{citacao}
``O exercício da obstetrícia tem oferecido à enfermagem uma valorização do seu corpo de conhecimentos técnicos e práticos, bem como uma expansão do mercado de trabalho, inclusive liberal, por meio da assistência ao parto domiciliar.'' \cite[p.69]{maia2010humanizaccao}. 
\end{citacao}

A profissionalização e o mercado de trabalho, na obstetrícia, tem o potencial de ser cada vez menos interessante para a medicina, ao mesmo tempo que cresce sua relevância dentro da enfermagem. No âmbito da formação e qualificação de enfermeiras obstetras, o MS passou a financiar centros de especialização em enfermagem obstétrica e cursos de aprimoramento para profissionais já tituladas. \cite{maia2010humanizaccao,massari2017contribuiccoes}.

Percebe-se, por meio das narrativas das participantes, que a sensibilização de outros profissionais da equipe pode ajudar a diminuir os riscos de violência obstétrica no CPN. Com isso, ratificamos o que foi dito por \citeauthor{maia2010humanizaccao} (\citeyear{maia2010humanizaccao}) em seu livro sobre o \textit{ethos} profissional do médico que se não ocorrer essa conscientização poderemos enfrentar sérias consequências ou já estamos enfrentados nos serviços obstétricos. Em contrapartida, a autora \citeauthor{massari2017contribuiccoes} (\citeyear{massari2017contribuiccoes}), afirma que a atuação das enfermeiras obstetras e/ou obstetrizes no CPN tem um impacto positivo na melhoria dos indicadores obstétricos.

\begin{citacao}
``Uma conscientização melhor dos médicos.'' (E-1)
\end{citacao}

\begin{citacao}
``Primeiro, a formação dos médicos, pra saberem o que é o parto e orientar eles, pois ainda tem muitos médicos antigos, que não aceitam. Nós temos aqui médicos excelentes, mas, que ainda tá para trás e não aceitam. Assim, tem muitos cursos voltados para Enfermagem, e o médico nunca está lá. Muitas orientações são passadas para a Enfermagem, mas não são passadas para o médico, então, isso deveria ser pra ambos, para gente ver se melhorava.'' (E-2)
\end{citacao}

\begin{citacao}
``Eu acho que o coordenador médico, conversar com os médicos, conversar e orientar melhor pra eles participarem mais, porque eu acho que eles não participam de reuniões daqui, é raríssimo, junto até com a Enfermagem, né? Mas não são todos, eu não tô generalizando. Eu digo assim, mas tem alguns que precisam realmente ser conversado.'' (E-3) 
\end{citacao}

Segundo MAIA, ARAÚJO e MAIA \citeyear{da2018violencia}, na sua pesquisa vem corroborar com os discursos das enfermeiras obstetras do CPN de Maracanaú que demonstra o processo de introdução de práticas humanizadas no processo de parturição e como médicos, e especialmente, médicas lidam com essa inovação na realidade do México. Em seu estudo, os autores ressaltam que os profissionais estudados desenvolveram três tipos de posturas, a saber: posturas abertas à introdução da humanização, as intermediarias, que envolviam a necessidade de negociação de posturas colaborativas e de cuidado no processo de parto, e posturas de resistência, os quais receavam introduzir práticas humanizadas em sua conduta. Destaca-se que a educação continuada de profissionais de saúde pode auxiliar no processo de humanização do parto normal.