\chapter{Resultados}
\label{chap:resultados}

Após a análise de forma criteriosa e detalhada dos estudos selecionados, identificando abordagens, objetos de estudo e relação com a temática, apenas os 06 estudos a seguir foram analisados. A seguir, segue-se a caracterização dos mesmos, evidenciando-se a base em que se encontravam, os autores do estudo, a cidade ou estado brasileiro em que foram desenvolvidos, a amostra, o interesse do estudo (objetivos), o tipo de estudo, local de coleta de dados, as ações desenvolvidas no pré-natal e por fim, estratégia que facilitasse a adesão ao tratamento. 	 	 	 	 	

\section{Assistência Pré-natal no contexto da estratégia de saúde da família \cite{dos2010assistencia}}

\begin{description}
\item \textbf{Base de dados:} BDENF-Enfermagem
\item \textbf{Autor:} Hélcia Carla dos Santos Pitombeira, Liana Mara Rocha Teles, Jamile de Souza Pacheco Paiva, Maysa Oliveira Rolim, Lydia Vieira Freitas, Ana Kelve de Castro Damasceno.
\item \textbf{Cidade/Estado/ País:} São Gonçalo do Amarante/Ceará/Brasil
\item \textbf{Amostra:} 632
\item \textbf{Interesse do Estudo:} Estudar o acompanhamento pré-natal oferecido no município de São Gonçalo do Amarante com base nas informações do SIAB, SISPRENATAL, SINASC, SINANNET e SIM.
\item \textbf{Tipo de Estudo:} Estudo descritivo, com abordagem quantitativa
\item \textbf{Local do Estudo:} SIAB, SISPRENATAL, SINASC, SINANNET e SIM
\item \textbf{Ações na consulta de pré-natal:} Exames VDRL
\item \textbf{Estratégia para adesão:} Busca ativa (realizada pelo ACS)
\end{description}

\section{Assistência pré-natal a gestante com diagnóstico de sífilis \cite{suto2016assistencia}}

\begin{description}
\item \textbf{Base de dados:} BDENF-Enfermagem
\item \textbf{Autor:} Cleuma Sueli Santos Suto, Débora Lima da Silva, Eliana do Sacramento de Almeida, Laura Emmanuela Lima Costa, Taiana Jambeiro Evangelista.
\item \textbf{Cidade/Estado/ País:} Jacobina/Bahia/Brasil
\item \textbf{Amostra:} 667 (SINAN e SisPreNatal) e 6 no estudo de campo (3 enfermeiras e 3 gestantes)
\item \textbf{Interesse do Estudo:} Investigar qual tem sido a assistência prestada às gestantes com sífilis na Atenção Básica.
\item \textbf{Tipo de Estudo:} Estudo transversal e exploratório
\item \textbf{Local do Estudo:} SINAN, SisPreNatal e nas unidades de Saúde da Família e residência das gestantes do município de Jacobina-Bahia.
\item \textbf{Ações na consulta de pré-natal:} Prescrição de penicilina G benzatina
\item \textbf{Estratégia para adesão:} Atividades educativas sobre sífilis na UBS
\end{description}

\section{Monitoramento das ações pró-redução da transmissão vertical da sífilis na rede pública do Distrito Federal \cite{tavares2012monitoramento}}

\begin{description}
\item \textbf{Base de dados}: BDENF-Enfermagem
\item \textbf{Autor}:     Leonor Henriette de Lannoy Coimbra Tavares, Onã Silva, Leidijany Costa Paz, Luís Antônio Bueno Lopes, Maria Liz Cunha de Oliveira, Maria Marta Lopes Macedo, Sônia Geraldes
\item \textbf{Cidade/Estado/ País}: Distrito Federal/Brasil
\item \textbf{Amostra}: 3.726
\item \textbf{Interesse do Estudo}: Analisar o perfil epidemiológico e a cobertura de realização do VDRL durante a gestação e o parto, em gestantes e parturientes atendidas na rede pública de saúde do Distrito Federal.
\item \textbf{Tipo de Estudo}: Epidemiológico
\item \textbf{Local do Estudo}: Hospital Regional da Asa Sul, Hospital Regional da Asa Norte, Hospital Regional da Ceilândia, Hospital Regional de Brazilândia, Hospital Regional de Taguatinga, Hospital Regional de Sobradinho, Hospital Regional de Planaltina, Hospital Regional de Samambaia, Hospital Regional do Gama,  Hospital Regional do Paranoá e Hospital Universitário de Brasília.
\item \textbf{Ações na consulta de pré-natal}: Preenchimento do cartão das gestantes incompleto, poucas gestantes apresentavam 3 VDRL no pré-natal. 
\item \textbf{Estratégia para adesão}: Não há
\end{description}

\section{Avaliação da assistência pré-natal em município do Sul do Brasil \cite{segatto2015evaluation}}

\begin{description}
\item \textbf{Base de dados}: BDENF-Enfermagem
\item \textbf{Autor}: Marília Judith Segatto, Suzinara Beatriz Soares de Lima, Marciane Kessler, Thais Dresch Eberhardt, Rhea Silvia de Avila Soares, Lidiana Batista Teixeira Dutra Silveira
\item \textbf{Cidade/Estado/ País}: Município de Segredo-Rio Grande do Sul do Brasil
\item \textbf{Amostra}: 80 gestantes.
\item \textbf{Interesse do Estudo}: Verificar a efetividade da assistência pré-natal por meio de indicadores de processo de um município da região Sul do Brasil.
\item \textbf{Tipo de Estudo}: Estudo descritivo, com delineamento documental, utilizando indicadores de processo da assistência pré-natal do período de janeiro a dezembro de 2011, por meio do Sistema de Acompanhamento do Programa de Humanização no Pré-Natal e Nascimento.
\item \textbf{Local do Estudo}: SISPRENATAL implantado no setor de Epidemiologia da Secretaria Municipal de Saúde.
\item \textbf{Ações na consulta de pré-natal}: VDRL, consultas, preenchimento inadequado do cartão da gestante
\item \textbf{Estratégia para adesão}: Não há
\end{description}

\section{Sífilis congênita no Ceará: análise epidemiológica de uma década \cite{da2013sifilis}}

\begin{description}
\item \textbf{Base de dados}: LILACS
\item \textbf{Autor}: Camila Chaves da Costa, Lydia Vieira Freitas, Deise Maria do Nascimento Sousa, Lara Leite de Oliveira, Ana Carolina Maria Araújo Chagas, Marcos Venícios de Oliveira Lopes, Ana Kelve de Castro Damasceno
\item \textbf{Cidade/Estado/ País}: Ceará/Brasil
\item \textbf{Amostra}: 10.000 nascidos vivos. 2.930 casos de sífilis congênita.
\item \textbf{Interesse do Estudo}: avaliar a taxa de notificação de sífilis congênita no estado do Ceará nos anos de 2000 a 2009 de acordo com a base de dados do Sistema Nacional de Agravos de Notificação- SINAN, descrever o perfil epidemiológico das gestantes cujos recém-nascidos tiveram sífilis congênita; verificar a realização do pré-natal e tratamento das gestantes cujos recém-nascidos tiveram sífilis congênita e a realização do tratamento de seus parceiros.
\item \textbf{Tipo de Estudo}: Estudo do tipo transversal e documental, com abordagem quantitativa.
\item \textbf{Local do Estudo}: Estudo documental, realizado em julho de 2010 a partir do banco de dados disponível no Núcleo de Informação e Análise em Saúde (NUIAS) da Secretaria de Saúde do Estado do Ceará (SESA- CE)
\item \textbf{Ações na consulta de pré-natal}: consultas realizadas de forma inadequada
\item \textbf{Estratégia para adesão}: Não há.
\end{description}

\section{Sífilis Na Gestação: Perspectivas e Condutas do Enfermeiro \cite{nunes2017sifilis}}

\begin{description}
\item \textbf{Base de dados}: BDENF-Enfermagem
\item \textbf{Autor}: Jacqueline Targino Nunes, Ana Caroline Viana Marinho, Rejane Marie Barbosa Davim, Gabriela Gonçalo de Oliveira Silva, Rayane Saraiva Felix, Milva Maria Figueiredo de Martino
\item \textbf{Cidade/Estado/ País}: Natal/Rio Grande do Norte/Brasil
\item \textbf{Amostra}: 4 enfermeiros 
\item \textbf{Interesse do Estudo}: Discutir as ações do enfermeiro na atenção pré-natal a gestantes com sífilis e identificar dificuldades encontradas pelos profissionais na adesão ao tratamento das gestantes e parceiros.
\item \textbf{Tipo de Estudo}: Estudo qualitativo, tipo descritivo-exploratório.
\item \textbf{Local do Estudo}: Unidade Mista Felipe Camarão
\item \textbf{Ações na consulta de pré-natal}: Interrogar sobre o estado da gestante, avaliar a efetividade e adesão ao tratamento dela e do parceiro. Orientar quanto à importância do tratamento, riscos para a sífilis congênita. Encaminhamento da gestante para tratamento medicamentoso. Identificar gestantes com sinal de alarme e/ou identificadas como de alto risco e encaminhá-las para consulta médica.
\item \textbf{Estratégia para adesão}: Informação a respeito do direito do teste mensalmente, solicitação mensal do VDRL. Acolher o paciente de forma que ela/ele se sinta segura(o) dando as devidas orientações. Desenvolver atividades educativas, individuais e em grupos. Orientar gestantes e a equipe quanto aos fatores de risco e à vulnerabilidade.
\end{description}

\chapter{Discussão}
\label{chap:discussao}

Observa-se que os estudos trazem a sífilis gestacional sob o enfoque de diferentes tipos de estudos. Entretanto, a abordagem quantitativa foi a mais adotada. A maioria dos trabalhos adotou como sua fonte de dados os sistemas de informação em especial o \acrshort{SINAN} (2) e o \acrshort{SISPRENATAL} (\acrlong{SISPRENATAL}) (3). Em se tratando dos estudos de abordagem qualitativa, houve um do tipo descritivo-exploratório que fez uso de entrevistas e análise de conteúdo.

Neste trabalho, optou-se por realizar uma revisão integrativa para entender melhor como se dá o atendimento à gestante com sífilis no pré-natal, deu-se especial atenção a trabalhos que abordam ações da enfermagem no combate a essa enfermidade, melhorando assim os indicadores referentes a sífilis congênita e a qualidade da assistência como um todo dentro do que se preconiza o \acrlong{PHPN} (\acrshort{PHPN}).

Os estudos tiveram como interesses principais identificar o perfil epidemiológico das gestantes com sífilis, avaliar com base nos dados de sistemas de informação se a assistência no pré-natal a pacientes com sífilis estava sendo realizada de modo adequado e com cobertura suficiente, apontar os problemas relacionados ao pré-natal de gestantes com sífilis, apontar quais atividades da consulta do enfermeiro e mesmo se houve subnotificação de casos de sífilis. 

O perfil identificado nos trabalhos revela que a maior parte das gestantes que realizaram as consultas de pré-natal eram pessoas socialmente excluídas com baixo nível de escolaridade, baixo nível socioeconômico e pouca idade. Existe ainda um estudo que aponta a questão racial como fator excludente ao acesso dos exames de prevenção e tratamento. \cite{dos2010assistencia,suto2016assistencia,tavares2012monitoramento,da2013sifilis}

Fica clara a importância do Estado na promoção da saúde através de iniciativas educacionais e comprometimento dos órgãos de gestão (secretarias de saúde) a estruturarem os serviços de saúde de modo a permitir a implementação dessas ações e atingir o público mais carente. Uma das formas de se promover isso é através da \acrlong{PNEP-SUS} (\acrshort{PNEP-SUS}). Essa política propõe metodologias e tecnologias voltadas para a promoção, proteção e recuperação da saúde pela diversidade de saberes, a ancestralidade, a produção de conhecimentos e a adição deles ao SUS \cite{vasconcelos1999educaccao}.

Já sobre a qualidade da assistência às pacientes no pré-natal, os estudos indicam uma ampla cobertura da \acrlong{ESF} (\acrshort{ESF}), que preconiza que no contexto do acompanhamento do pré-natal existe uma quantidade mínima de consultas, acesso aos exames - como o teste rápido para sífilis, \acrshort{HIV} e \acrshort{VDRL} em dois momentos do pré-natal (primeiro trimestre e por volta da trigésima semana) - além de garantir fácil acesso ao tratamento.

Apesar de existir um número satisfatório de equipes de saúde da família para atender a demanda, fica constatado nos trabalhos que isso não foi suficiente. De fato, foram identificadas situações onde o início no pré-natal nem sempre se deu no primeiro trimestre e os exames ou não eram realizados ou eram realizados em menor quantidade do que é preconizado, ou ainda ficavam sem o devido registro. Em alguns dos estudos, isso era atribuído ao baixo conhecimento do profissional sobre o manejo correto da gestante com sífilis e dos instrumentos para registro e alimentação do sistema de notificação. \cite{tavares2012monitoramento,da2013sifilis,da2013sifilis}

Em relação aos problemas relacionados ao pré-natal  de gestantes com sífilis, apesar da cobertura no acompanhamento ser adequada, os estudos mostram que isso não garante a qualidade da assistência a essas mulheres \cite{tavares2012monitoramento,da2013sifilis,da2013sifilis}. Em diversos locais e contextos, a consulta de pré-natal não contemplou as ações indicadas como boas práticas como constam no \acrlong{PHPN} (\acrshort{PHPN}). Dentre as ações que não foram realizadas ou foram realizadas inadequadamente, destacam-se: 

\begin{enumerate}
\item A realização do teste rápido e dos exames de \acrshort{VDRL} nos momentos corretos;
\item O tratamento com penicilina G benzatina nas dosagens adequadas; 
\item O rastreamento e tratamento dos parceiros;
\item Falha nos registros (cartão da gestante) e na notificação dos casos de sífilis em gestantes através do \acrshort{SINAN}.
\end{enumerate}

Nos trabalhos selecionados, apenas um abordou estratégias e ações para aumentar a adesão ao tratamento de sífilis, trabalho realizado no Rio Grande do Norte com quatro enfermeiros (Estudo 6). Nele foram identificadas estratégias como: 

\begin{enumerate}
\item Conscientização das pacientes ao direito de realizar novos exames de VDRL mensalmente;
\item Acolhimento com segurança, humanização e informação a respeito do seu estado de saúde assim como dos riscos que a sífilis traz ao feto;
\item Desenvolver atividades educativas individuais e coletivas.
\end{enumerate}

Já no que diz respeito às ações, foram identificadas as seguintes: 

\begin{enumerate}
\item Contínuo monitoramento das pacientes e adesão ao tratamento; 
\item Orientação sobre a importância de realizar a terapêutica dela e do parceiro;
\item Orientar como identificar sinais de alerta e caso eles ocorram realizar o encaminhamento para consulta médica. 
\end{enumerate}

Nota-se assim, a carência de estudos qualitativos que aprofundem a temática sugerida para esta revisão, uma vez que pesquisas quantitativas, baseadas em dados de sistemas de informação, podem ficar restritas, não abarcando particularidades dos diversos contextos em que se dá esse acompanhamento. 

O pré-natal se faz além da clínica e do segmento de protocolos, parte também do vínculo e do que ele promove entre os sujeitos que formam essa ponte. É preciso maior investigação a respeito das subjetividades que envolvem os atores desse cuidado. 
