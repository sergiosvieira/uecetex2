\chapter{Considerações Finais}
\label{chap:conclusoes-e-trabalhos-futuros}

As enfermeiras obstetras atuantes no CPN revelaram com propriedade suas percepções acerca da violência obstétrica, no entanto ainda encontramos algumas limitações devido ao novo olhar para o atendimento dessas profissionais no CPN, pois a implementação desse equipamento de saúde ainda é recente. O tema em estudo é complexo e envolve uma rede de elementos, tais como a necessidade de inserção do tema humanização no cotidiano das ações realizadas entre profissionais de saúde e parturiente, formação em saúde, educação permanente, desses profissionais e dos serviços especializados em obstetrícia, principalmente no CPN.

Diante disso, ratificamos a necessidade de políticas públicas eficazes no combate a este tipo de violência. Destaca-se a necessidade de inclusão do termo violência obstétrica nos descritores, preferencialmente em documentos legais que a definam e a criminalizem. Assim, poderá auxiliar na identificação e enfrentamento dessas situações. Também, podemos ressaltar o papel das enfermeiras obstetras na redução desses casos no CPN, promovendo, assim, assistência pautada em princípios como a integralidade e a equidade. 

Há demanda para novos estudos que integrem informações e pesquisas, possibilitando visibilidade as produções relativas ao tema utilizando metodologias que contemplem as abordagens mistas devido à complexidade da temática no processo de construção do conhecimento acerca da violência obstétrica pelos profissionais da saúde.
