\chapter{Conclusões}
\label{chap:conclusoes-e-trabalhos-futuros}

Apesar de existir uma terapêutica de baixo custo e de uma cobertura adequada pelas equipes da \acrshort{ESF}, muitas vezes o tratamento para a sífilis na gestação não é efetivo, visto que existem algumas dificuldades na adesão e efetividade do tratamento que fragilizam a prevenção que estão intimamente relacionados à assistência pré-natal.

Também identificou-se situações onde o enfermeiro declara não se sentir confiante ao lidar com casos de sífilis, além de relatar a falta de conhecimento acerca dos documentos necessários para a notificação dos casos. Outras dificuldades relatadas foram: ausência da realização e atraso na entrega dos exames; abandono de pré-natal; dificuldade na captação e tratamento do parceiro; além da presença de dados incompletos nos prontuários e fichas epidemiológicas.

Em nota informativa de 2017 do \acrlong{DIAHV} (\acrshort{DIAHV}) houve uma mudança na definição de caso de sífilis congênita, deixando-se de considerar o tratamento da parceria sexual da mãe, proporcionando assim adequação da sensibilidade na captação de casos. E no caso das gestantes, toda mulher que for diagnósticada com sífilis seja no pré-natal, parto ou puerpério serão notificadas como sífilis em gestante ao invés de adquirida. Portanto, a partir de 2017, é provável que o incremento observado nos casos de sífilis em gestantes possa ser atribuído, em parte, à mudança no critério de definição de caso. \cite{boletim2018}

Baseada nestas mudanças, propõe-se novos estudos a fim de averiguar a realização da notificação adequada nas unidades que prestam atendimento às gestantes, evitando assim, subnotificações e um direcionamento adequado de políticas públicas para melhoria nos indicadores. 
			
É mostrado nos estudos a ocorrência de falhas no processo de notificação, além de problemas quanto a execução do tratamento. Atividades de educação em saúde e as orientações às pacientes podem não estar tendo o efeito esperado ou não são realizadas, haja vista o número crescente de gestantes jovens com sífilis e a alta incidência de sífilis congênita. É importante também ressaltar que a formação continuada voltada para o manejo dessa população é extremamente necessária para reduzir as falhas nos processos envolvidos no pré-natal a essa mulher. Fazem-se necessárias estratégias que facilitem a participação do parceiro no pré-natal, promovendo assim um maior vínculo e adesão deste a realização de testes e tratamento para evitar a reinfecção da mulher e minorar os riscos de transmissão vertical.
 
Os trabalhos apresentados nesta revisão mostram que a o pré-natal a gestante com sífilis ainda é precário para a resolução desse problema e a melhora nos indicadores de sífilis congênita. Apesar dos estudos quantitativos trazerem esses dados com clareza, eles delimitam a compreensão a respeito das particularidades que levam a esses resultados. Estudos de abordagem qualitativa, como o do Rio Grande do Norte, exploram discursos dos sujeitos envolvidos, revelando dados que podem responder as reais causas para essas falhas no atendimento, bem como soluções que podem ser multiplicadas e adaptadas a diversas realidades.  

Com base na quantidade de estudos encontrados não podemos afirmar que os enfermeiros não desenvolvem adequadamente as ações de pré-natal junto às gestantes e seus parceiros. O que foi identificado é que faltam estudos realizados em campo que tragam essa temática, de forma a elucidar quais as barreiras para o manejo adequado da gestante diagnosticada com sífilis e como superá-las. Fazem-se necessários estudos  a respeito de ações que tenham impacto na mitigação desse problema de saúde, podendo ser adaptadas e multiplicadas para diversos cenários. 
