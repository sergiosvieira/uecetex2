\chapter{Conclusões e Trabalhos Futuros}
\label{chap:conclusoes-e-trabalhos-futuros}

O presente estudo retratou que na análise da rede social de uma enfermeira da ESF do município de Icapuí para atender às demandas de paciente hipertensos e diabéticos, emergiram atores que em sua maioria eram enfermeiros ou profissionais da enfermagem, identificando o potencial do enfermeiro para estabelecer a comunicação entre outros trabalhadores da saúde e diferentes setores e instituições de saúde do município. 

Além disso, através dos discursos podemos perceber os vínculos pessoais (informais) e profissionais (formais) que influenciam a qualidade desses laços e em como eles possibilitam a expansão da rede para diferentes níveis de atenção, criando articulações que fogem ao sistema convencional de contato, mas que é eficiente na resolução dos problemas, pois conta com a cooperação entre os atores.

A técnica de Análise de Redes Sociais permitiu, com o desenho do grafos e o uso das medidas de centralidade, desvendar características dessas ligações, revelando os atores mais importantes nesse retrato para garantir as ações voltadas para a continuidade do cuidado a hipertensos e diabéticos do município sob a perspectiva dessa rede. 

Entretanto, por se tratar da análise de uma única rede, apresenta limitações a respeito da elucidação de toda a rede municipal para esses pacientes, fazendo-se necessária, para um retrato mais completo, estudos futuros que envolvam as demais equipes da ESF do município. 

A entrevista semi-estruturada apresentou-se como instrumento satisfatório para a coleta de dados, pois enriqueceu os achados qualitativos dos grafos e das medidas revelando aspectos a respeito do vínculo dos profissionais entre si que originavam e fortaleciam os laços. 




