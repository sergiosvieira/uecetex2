\chapter{Introdução}
\label{cap:introducao}

As infecções acometidas por via sexual aumentaram substancialmente nos últimos anos no Brasil e no mundo. Segundo dados da \acrlong{OMS} (\acrshort{OMS}), ocorram em torno de um milhão de casos de \acrlong{IST} (\acrshort{IST}) por dia. Dentre as principais, destacam-se: clamídia, gonorreia, sífilis e tricomoníase. No período de gestação, a sífilis acomete a mais de 300.000 mortes fetais e neonatais por ano no mundo e potencializa o risco de morte prematura em outras 215.000 crianças. \cite{boletim2018}

A sífilis é uma doença infeciosa que evolui cronicamente apresentando períodos agudos e de latência. Ela tem como agente etiológico o Treponema pallidum, uma bactéria gram negativa com forma espiral do grupo das espiroquetas \cite{mendes2005microbiologia}, que pode ser transmitida pelo ato sexual ou por transmissão vertical, ocasionando, neste caso, a \acrlong{SC} (\acrshort{SC}) no feto. A sífilis possui simples diagnóstico e tratamento, sendo esse de baixo custo. \cite{brasilprenatal}

Por estar no rol das \acrshort{IST}s, a referida doença traz grande preocupação no mundo pela sua fácil transmissão tanto de forma horizontal como vertical e danos causados ao paciente a médio e longo prazo, sobretudo durante a gestação, ela pode acarretar problemas tanto para a mãe como para o feto. A sífilis é uma infecção que requer maior atenção sobretudo no período gestacional, verticalmente transmissível, podendo causar infecções congênitas de graus variáveis com a idade fetal, determinando teratogenias ou doenças crônicas graves, podendo levar a morte fetal ou perinatal que pode ser eliminada, quando identificada e tratada, seja antes ou durante a gestação. \cite{avelleira2006}

De acordo com \cite{brasilprenatal}, dentre as possíveis complicações ocasionadas pela sífilis, além de um provável nascimento de crianças já infectadas e com sintomas da doença, estão o abortamento tardio, natimortos, hidropsia fetal e parto prematuro \cite{brasilprenatal}. A fim de evitar-se tais desfechos negativos, a prevenção, a detecção precoce, o diagnóstico e o tratamento durante o pré-natal fazem-se necessários.

Na última década, no Brasil, observou-se um significativo aumento de notificações de casos de sífilis adquirida, sífilis em gestantes e sífilis congênita, que pode ser condicionado, em parte, ao aprimoramento e incremento de políticas públicas, como o aperfeiçoamento do sistema de vigilância e à ampliação da utilização de testes rápidos. \cite{boletim2018}

No ano de 2017, foram notificados no \acrlong{SINAN} (\acrshort{SINAN}) 119.800 casos de sífilis adquirida (taxa de detecção de 58,1 casos/100 mil habitantes); 49.013 casos de sífilis em gestantes (taxa de detecção de 17,2/1.000 nascidos vivos); 24.666 casos de sífilis congênita (taxa de incidência de 8,6/1.000 nascidos vivos); e 206 óbitos por sífilis congênita (taxa de mortalidade de 7,2/100 mil nascidos vivos). \cite{boletim2018}

O Departamento de Vigilância, Prevenção e Controle das \acrshort{IST}, do HIV/AIDS e das Hepatites Virais, vinculado à \acrlong{SVSMS} (DIAHV/\acrshort{SVSMS}) recomenda que se institua Comitês de Investigação para Prevenção da Transmissão Vertical em municípios, estados ou regiões que tiverem altos índices de casos de sífilis congênita, com o intuito de observar os possíveis problemas que possam ocasionar a transmissão vertical da sífilis e sugerir medidas que demonstrem resolutividade na prevenção, diagnóstico, assistência, tratamento e vigilância do agravo. Além de que se deva mensurar a capacidade local de melhorar os Comitês de Prevenção de Mortalidade Materna, Infantil e Fetal (ou de outros comitês/grupos existentes) para somar a discussão de casos de transmissão vertical, considerando essa mesma finalidade. \cite{boletim2018}

Uma nota de relevância para o referido estudo é que para fins de vigilância epidemiológica, os critérios para definir os casos de sífilis adquirida, sífilis em gestantes e sífilis congênita foram alterados em 2017 por meio da Nota Informativa número 2 SEI/2017  DIAHV/SVS/MS, a fim de se chegar a um padrão adequado na captação de casos de sífilis congênita e diminuir a subnotificação de casos de sífilis em gestantes. Logo, na definição de caso de sífilis congênita, não é mais considerado o tratamento da parceria sexual da mãe; e no caso de sífilis em gestantes, deliberou-se que todas as pacientes diagnosticadas com sífilis durante o pré-natal, parto e/ou puerpério devem ser notificadas como caso de sífilis em gestantes, e não como sífilis adquirida. \cite{boletim2018}

A prevenção pelo profissional de saúde é feita durante as consultas de pré-natal com orientações e informações a respeito da sífilis e de outras ISTs, bem como o uso de preservativo nas relações sexuais. A detecção acontece ainda na primeira consulta com a realização de testes-rápidos não apenas para a sífilis, mas também para HIV \cite{brasilprenatal}. Utilizar preservativo como forma de prevenção ainda é essencial. Entretanto, outras intervenções são também eficazes e deveriam ser adotadas conjuntamente à proposta de prevenção. \cite{brasil2015protocolo}

Os pesquisadores \cite{bagatini2016teste} em um estudo concluíram que os testes rápidos para HIV e sífilis, realizados pelos profissionais de referência no próprio território da população, possibilitam um diagnóstico mais rápido para HIV ou rastreamento de sífilis, pois como a população não precisa se deslocar a grandes distâncias, tornando-se mais fácil um projeto terapêutico que abordem tanto o tratamento quanto o segmento do paciente.

É de suma importância que uma vez positivado o teste-rápido, tanto gestante quanto o parceiro inicie a terapêutica medicamentosa, feita com Penicilina Benzatina, para se evitar a transmissão vertical e também a reinfecção da gestante. Segundo \cite{brasil2015protocolo} a perfeita combinação entre diagnóstico precoce e tratamento realizado adequadamente e oportunamente não apenas da sífilis, mas também de outras ISTs durante a gestação, previne a transmissão vertical, devendo receber a devida atenção em todos os níveis de assistência componentes do SUS.

Apesar de uma terapêutica barata, muitas vezes o tratamento para a sífilis na gestação não é efetivo, visto que existem algumas dificuldades na adesão e efetividade do tratamento que fragilizam a prevenção da referida infecção. Estas dificuldades estão intimamente relacionados à assistência pré-natal e são elas: ausência da realização e atraso na entrega dos exames; abandono de pré-natal; falta de captação e resgate das gestantes faltosas; dificuldade no manejo da infecção por parte dos profissionais; dificuldade na captação e tratamento do parceiro; falta de seguimento das mães e crianças após o parto; além da presença de dados incompletos nos prontuários e fichas epidemiológicas, segundo \cite{cardoso2018analise}.

No Brasil, doenças infecciosas, como a do objeto de estudo, durante a gravidez são frequentes de maneira relativa. A transmissão para o bebê pode se dar durante a gestação, no parto em si ou durante o aleitamento materno. Há uma estimativa de que 40\% das mulheres grávidas com sífilis primária ou secundária não tratada evoluem para perda do concepto. E que uma porcentagem superior a 50\% dos recém-nascidos filhos de mães infectadas não tratadas ou tratadas de forma não efetiva, não manifestam sintomas da doença, que pode levar assim a não ser diagnosticados ao nascerem, com sérias consequências no futuro, conforme \cite{rodrigues2016elementos}.

Na \acrlong{ABS} (\acrshort{ABS}), o enfermeiro está habilitado a realizar conjuntamente com o médico as consultas de pré-natal de risco habitual ou baixo risco. Por vezes esse profissional se depara com uma situação que requer seu conhecimento não apenas técnico, mas também social e humano.

A consulta de pré-natal pela enfermagem pode ser uma ferramenta indispensável se realizada adequadamente, pois planeja o cuidado, que pode ser desenvolvido de forma integral e humanizada. Uma consulta assim envolve não apenas aspectos técnicos-científicos como também foca nas necessidades relacionais e emocionais das mulheres, envolvendo também os familiares, criando vínculos e estimulando o auto-cuidado e a responsabilização de todos os envolvidos. É uma consulta pautada no diálogo e na confiança, em ações efetivas e continuadas. \cite{rodrigues2016elementos}

Assim, diante da persistência da sífilis na gestação como um grave problema de saúde pública e da sífilis gestacional como indicador da qualidade do pré-natal, e estando o enfermeiro em uma posição de destaque na consulta de pré-natal, questiona-se com esse estudo quais as produções científicas sobre a consulta de enfermagem no pré-natal a gestante com sífilis de modo a aumentar a adesão ao tratamento das mesmas?

A escolha pelo tema teve origem com a vivência durante o desenvolvimento do componente prático do curso de especialização em Enfermagem Obstétrica. Os locais para campo de prática foram um hospital-escola da rede municipal de Fortaleza que é referência em serviços obstétricos, e uma Unidade de Atenção Primária à Saúde (\acrshort{UAPS}). Na rotina do hospital eram realizados testes-rápidos a todas as parturientes, além da avaliação de presença ou não de registro na Caderneta da Gestante a respeito da realização do mesmo teste durante o pré-natal e dos exames de VDRL (\acrlong{VDRL}).

Percebeu-se que uma quantidade considerável de parturientes chegava ao serviço hospitalar sem conhecimento a respeito da doença, seus riscos e possíveis desfechos negativos sobre sua saúde e sobre a da criança, formas de contágio e prevenção. As que apresentavam teste-rápido realizado no hospital com resultado positivo, eram indagadas sobre seu pré-natal, possíveis tratamentos realizados e eram checados os dados das cadernetas. 

Notou-se que muitas mulheres tinham seu cartão de pré-natal preenchido adequadamente, mas tinham pouco conhecimento sobre a sífilis e outras \acrshort{IST}. Algumas não realizaram ou não tinham registro de seus testes-rápidos e VDRL no pré-natal. Outras ainda, iniciaram algum tipo de antibioticoterapia, mas sem a participação do parceiro no esquema terapêutico ou com abandono do tratamento de um ou de ambos.

A pesquisadora também desenvolve atividades em educação em saúde sexual e reprodutiva em uma \acrshort{ONG} do município de Caucaia-Ce e tem a percepção que as mulheres em idade fértil, no geral ainda apresentam muitas lacunas no conhecimento a respeito das IST, deixando muito a cargo do parceiro a escolha de métodos prevenção. Falar sobre saúde sexual e reprodutiva e do protagonismo da mulher nessa questão ainda é um tabu, o que dificulta essa autonomia, tomada de decisão e maior adesão às ações terapêuticas e preventivas.

Uma vez exposta toda a problemática da sífilis na gestação, o estudo se mostra relevante, pois procura levantar de uma forma geral, o que existe de produção científica sobre a temática, podendo sugerir novas produções científicas, especialmente partindo da enfermagem e abordando a Atenção Básica como campo para estudo. 

As gestantes e suas famílias podem ser beneficiadas, pois, o conhecimento gerado pode indiretamente suscitar mudanças na abordagem às mulheres durante a consulta pré-natal, por parte da enfermagem, atendendo a demandas das mães de uma forma mais holística.

Além disso, o presente estudo serve aos gestores como passo inicial na investigação de como as políticas públicas podem se reorganizar de forma a sanar as deficiências do sistema através de educação continuada para os profissionais a fim de melhorar abordagem a essa população, melhorando assim os indicadores.

Com os resultados e discussões dessa pesquisa os profissionais de enfermagem poderão refletir sua prática ao atender essa clientela de forma a aprimorarem seus conhecimentos e estratégias para tornar a consulta de pré-natal a essas mulheres um momento de cuidado que vai além do protocolo, pois requer o desenvolvimento de habilidades sociais e pessoais para a criação de vínculos e participação ativa dos sujeitos no seu processo terapêutico.




