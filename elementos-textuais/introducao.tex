\chapter{Introdução}
\label{cap:introducao}

A violência pode ser definida como a imposição de um grau significativo de força física ou intimidação moral e sofrimentos que podem ser evitáveis; especificamente, a violência obstétrica contra a mulher abrange a violência física, sexual e/ou psicológica. \cite{cardoso2017violencia}

Nesse ínterim, o trabalho de parto e parto é visto e tratado apenas na sua forma mecânica sem nenhuma empatia com a parturiente, em que os profissionais da área da saúde deixam a mesma em um segundo plano no momento do parto. Assim ela passa a ser apenas mais um componente daquele ato. \cite{cardoso2017violencia}

No mundo, mulheres sofrem abusos, desrespeito e maus-tratos durante o parto nas instituições de saúde. Tal tratamento não apenas viola os direitos das mulheres ao cuidado respeitoso, mas também ameaça o direito à vida, à saúde, à integridade física e a não discriminação. \cite{cardoso2017violencia}

“Entende-se por violência obstétrica qualquer ação promovida pelos profissionais da saúde no que diz respeito ao corpo e aos processos reprodutivos da mulher, caracterizando-se por uma assistência desumanizada, abuso de ações intervencionistas, medicalização e reversão do processo de parto de natural para patológica”. (CARDOSO et al., 2017, p.3347).


Em 1950, nos EUA, ocorreram os primeiros relatos de violência obstétrica, quando a Ladies Home Journal, uma revista para donas de casa, publicou a matéria “Crueldade nas Maternidades”. O texto descrevia como tortura o tratamento recebido pelas parturientes, submetidas ao sono crepuscular (twilight sleep), uma combinação de morfina e escopolamina, que produzia sedação profunda, não raramente acompanhada de agitação psicomotora e eventuais alucinações. (DINIZ, et al., 2015).

Na década de 1970 no Brasil, iniciou-se o combate contra a violência obstétrica com o Movimento pela Humanização do Parto, mas essa causa só tomou amplitude significativa em 1993, com a criação da Rede pela Humanização do Parto e do Nascimento (ReHuNa), congregando a participação de profissionais e instituições em torno de uma assistência obstétrica voltada para o respeito à fisiologia da mulher gestante ou parturiente e o bebê. (SOARES, 2017.)

No Brasil, ao final da década de 1980 a violência obstétrica já era tema nas políticas de saúde. O Programa Assistência Integral à Saúde da Mulher (PAISM), por exemplo, que incorporou o ideal feminista para atenção à saúde da mulher integral, inclusive responsabilizando o Governo com os aspectos da saúde reprodutiva. Além disso, propôs uma forma mais humanizada na relação entre profissionais da saúde e as mulheres, apontando sua autonomia em seu corpo e um maior controle sobre sua saúde. Porém, ainda que o tema estivesse na pauta feminista e mesmo na de políticas públicas, foi relativamente negligenciado, diante da resistência dos profissionais e de outras questões urgentes na agenda dos movimentos, e do problema da falta de acesso das mulheres pobres a serviços essenciais. (DINIZ, et al., 2015).

A violência obstétrica ganhou visibilidade, na metade do século XXI, sendo tema de vários estudos, mostras artísticas, documentários, ação no judiciário, investigação parlamentar, atuações de diversas instâncias do Ministério Público (MP), assim como de um novo conjunto de intervenções de saúde pública. (DINIZ, et al., 2015).

Esses movimentos atuam em defesa da mulher como real protagonista do parto, creditando sua capacidade de conduzir este evento fisiológico, o direito de a assistência de qualidade e respeito durante todo o processo e de fazerem suas escolhas após receberem informações verídicas sobre o parto e os procedimentos médicos possíveis. (SOARES, 2017).

Entretanto, no Brasil, 1 em cada 4 mulheres diz ter sofrido algum tipo de violência obstétrica e o país assistiu nas últimas décadas um crescimento alarmante do índice de cesáreas, segundo informações do Departamento de Informática do Sistema Único de Saúde – DATASUS, de 2015, os partos hospitalares representam 98,08% dos partos realizados na rede de saúde e, entre os anos de 2007 e 2011, houve um aumento de 46,56% para 53,88% de partos cesáreas. (SOARES, 2017; ZANARDO et al., 2017).

De acordo ZANARDO et al., (2017), esses números de cesarianas variam entre o atendimento nos sistemas público e privado de saúde, que apresentam uma ocorrência de aproximadamente 40% e 85%, respectivamente. Esse cenário é considerado preocupante quando se leva em conta que a recomendação da Organização Mundial da Saúde – OMS (World Health Organization, 1996) é de uma taxa de cesáreas que varie entre 10 a 15%. Essa recomendação está baseada em estudos que apontam que uma taxa maior que 15% não representa redução na mortalidade materna e tampouco melhores desfechos de saúde para o binômio mãe-bebê. 

Estes dados refletem a situação de desrespeito aos direitos humanos das mulheres, que foram e continuam sendo submetidas as cesárias desnecessárias, sem fundamentos coerentes sobre a necessidade dos procedimentos e a omissão dos seus riscos e complicações. (SOARES, 2017.)

Nessa perspectiva, salienta-se a pesquisa nascer no Brasil, realizada entre 2011e 2012, que teve como um de seus objetivos analisar as intervenções obstétricas em mulheres em trabalho de parto de risco habitual. Esse estudo nacional de base hospitalar, composto por puérperas e seus recém-nascidos das diferentes regiões do país, revelou que, da amostra total da pesquisa de 23.940 mulheres, 56,8% foram consideradas como casos de risco obstétrico habitual, ou seja, sem condições de saúde que indicassem o uso de procedimentos e intervenção cirúrgica. Dentre essas mulheres, 45,5% realizaram cesárea e 54,5% tiveram parto vaginal, porém, apenas 5,6% tiveram parto normal sem nenhuma intervenção (ZANARDO et al., 2017). 

Em estudo realizado em 2011, na Venezuela, foi observado que as principais infrações à Lei Orgânica sobre o Direito das Mulheres a uma Vida Livre de Violência aconteceram por meio de tratamento desumano em 66,8% dos casos (21,6% dos quais em razão das críticas aos gritos durante o parto; 19,5% pela proibição de perguntar algo e/ou manifestar seus medos e inquietudes; 15,3% pelas piadas acerca da sua condição, com comentários irônicos e desclassificatórios) e a realização de procedimentos médicos sem consentimento prévio em 49,4% dos casos (dos quais 37,2% pela realização de toques vaginais repetitivos e por múltiplos examinadores). (RODRIGUES et al., 2015).

Nos últimos anos, organizações e ações voltadas aos Direitos Humanos e das Mulheres, vêm elaborando projetos que chamem atenção da Legislação Brasileira para que medidas sejam adotadas a fim de minimizar e erradicar a incidência crescente de violência obstétrica no país. Dentre essas medidas estão à valorização do parto normal e propondo soluções mais humanizadas aos ambientes altamente medicalizados e a criação do Centro de Parto Normal (CPN) que é uma unidade de atendimento ao parto de baixo risco sem distócia. (SOARES, 2017; BITENCOURT, BARROSO-KRAUSE, 2004).

O CPN foi formalmente criado através da publicação de uma Resolução do Ministério da Saúde (Portaria no 985, de 5 de agosto de 1999), que estabeleceu os parâmetros legais para sua implantação. Tal formalização propunha-se a atender à urgente necessidade da atenuação dos óbitos maternos por causas evitáveis, garantia do acesso ao parto em serviços de saúde de forma universal e a expansão dessa cobertura. Esta denominação de CPN, logo passaria a ter a sua referência de nomenclatura reconhecida como “Casa de Parto”.  (BITENCOURT, BARROSO-KRAUSE, 2004).

Diante disso, surgiu um questionamento: Quais as percepções do enfermeiro sobre violência obstétrica no Centro de Parto Normal (CPN)? Justifica-se a realização dessa pesquisa por motivo da insatisfação da sociedade acerca da conduta de profissionais da saúde cometendo violência obstétrica contra parturientes nos serviços obstétricos no país. Salienta-se a necessidade de conhecer esta realidade a partir de profissionais do CPN, local constituído com a premissa de naturalizar/ humanizar o parto e evitar condutas violentas. 

            Diante disso, ao terminar a graduação em Enfermagem, iniciei os estudos na pós-graduação em Enfermagem Obstétrica para poder compreender mais sobre esse assunto, e assim desenvolver um trabalho com o intuito de melhorar assistência da gestante e a qualidade do trabalho de parto e parto.

Dessa maneira, ao enfatizar esse assunto, trazemos à baila a relevância de discutir a temática, pois ainda existem muitos casos de violência obstétrica acontecendo nos dias atuais. Com isso, produziremos conteúdos nessa pesquisa que poderão ajudar a reflexão dos profissionais no acompanhamento do trabalho de parto e parto sem necessidade de maltratar a mulher. 

Ressalta-se que a presente pesquisa faz parte do projeto “Gestão e Cuidado no Centro de Parto Normal: Desafios como política pública”, coordenado pelo Prof. Dr. Antonio Rodrigues Ferreira Júnior.
