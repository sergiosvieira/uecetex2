\chapter{Introdução}
\label{cap:introducao}

O sistema de saúde brasileiro constitui-se de níveis de atenção: primário, secundário e terciário. Assim distribuídos de acordo com seu padrão de complexidade. O enfermeiro está presente em todos os níveis de atenção. Entretanto, na \acrshort{APS} (\acrlong{APS}) suas práticas podem ser de caráter mais ampliado, pois como um nível que preza a prevenção de doenças e/ou agravos e a promoção à saúde, a APS proporciona a esse profissional uma maior autonomia no cuidado ao indivíduo e à coletividade. 

Segundo \cite{neves2011manual}, enquanto em muitos países o centro do trabalho na Atenção Primária está no médico, no Brasil as políticas públicas estão propondo uma equipe multidisciplinar para atuar no que aqui por questões de construção histórica de lutas e resistências, convencionou-se chamar de Atenção Básica.

Para isso, o enfermeiro deve se utilizar de outros saberes, sobretudo ligados às ciências sociais e humanas, rompendo com o ainda empoderado modelo biomédico, centrado na doença e nas práticas terapêuticas convencionais em que a figura central é o profissional médico. Esse novo cenário encontra espaço na \acrshort{ESF} (\acrlong{ESF}), onde as equipes de saúde devem atuar de forma multi e interdisciplinar provendo assim um cuidado integral no âmbito individual e coletivo. 

Para \cite{guedes2007tripe}, a fim de reorientar as práticas de saúde para o alcance dos princípios do \acrshort{SUS} (\acrlong{SUS}), a \acrshort{ESF}, surge para conduzir ações para a promoção da saúde. Os princípios da promoção da saúde permitem a construção do cuidado por meio de trocas solidárias, críticas, capazes de envolver a comunidade, desenvolver as habilidades pessoais, criar ambientes saudáveis e reorganizar os serviços de saúde, integrando as instâncias práticas da vida das pessoas ao seu estado de saúde.

O enfermeiro tem importante papel dentro da equipe, pois entre as suas atribuições tem destaque a consulta de enfermagem dentro dos mais variados programas institucionalizados pelo Ministério da Saúde para populações com demandas específicas, como por exemplo a consulta de HIPERDIA para hipertensos e diabéticos, o Pré-natal, o Planejamento Familiar, o programa de Imunização das diversas faixas etárias, entre outros além das visitas domiciliares em equipe, reuniões com os \acrshort{ACS} (\acrlong{ACS}) e atenção aos casos de hanseníase, tuberculose, raiva humana  além da colaboração com outros profissionais da equipe.  Para \cite{lunardi1994enfermeiro}, o enfermeiro pode ser considerado um facilitador do trabalho dos demais profissionais, tanto da equipe de enfermagem como de outras áreas.

Assim, o cuidado produzido pelo enfermeiro, seja ele gerencial ou clínico, individual ou coletivo, envolve também o acesso e a integralidade na Atenção Básica, que por sua vez validam sua importância dentro da equipe de saúde, mesmo que em meio às incertezas na gestão ou a conflitos dentro da equipe. \cite{matumoto2011pratica, weirich2009trabalho, heringer2007praticas, david2009organizaccao}

Exercer essas funções de forma satisfatória requer um conhecimento do território e de sua população adscrita. Isso é facilitado pela atuação dos \acrshort{ACS}, que imersos no território e membros da comunidade assistida, criam uma ponte entre a \acrshort{UBS} (\acrlong{UBS}) e as equipes que a compõem. A comunicação, as pactuações e os laços fortes são formados nesse universo e uma vez que geralmente é o enfermeiro o primeiro contato do usuário com os serviços de saúde, cabe a ele desempenhar o papel de articulador do cuidado respeitando a realidade social, econômica, política e cultural da comunidade, fazendo-se cumprir os princípios do \acrshort{SUS} de integralidade, universalidade e equidade. 

Esses vínculos estabelecidos acabam se estruturando em verdadeiras redes. Redes essas que envolvem os sujeitos, membros da equipe de saúde, comunidade e atores sociais de outros setores de atenção, que devidamente conectados com fortes vínculos, proporcionam apoio aos indivíduos que as utilizam e meios para a resposta dos serviços públicos às demandas que surgem. 

As redes sociais e apoio social que estas podem gerar são cada vez mais objetos de estudos nas ciências da saúde. Haja vista que o enfermeiro encontra-se muitas vezes imbricado dentro dessas redes e partindo-se do princípio que seu papel dentro delas, faz-se necessário conhecer seus significados e o papel do enfermeiro dentro delas na Atenção Básica à Saúde. 

Para \cite{marteleto2009informaccao} o conceito de redes sociais é recentemente tratado no campo da Saúde Coletiva, entretanto seu uso vem crescendo. É um termo que possui vários significados mas, para alguns autores, tem o conceito de junção. Uma ideia defendida a partir de uma perspectiva histórica, tomando-o como uma ferramenta cognitiva que busca compreender as conjunturas e situações complexas, admitindo um olhar interdisciplinar.

Reconhecendo o enfermeiro como primeiro contato do usuário do serviço de saúde na \acrshort{ABS}, sua autonomia gerencial do cuidado na \acrshort{ESF} e seu potencial como articulador de uma atenção integral sendo e fazendo uso de conexões dentro das redes sociais formadas, questiona-se: qual o papel do Enfermeiro na articulação das Redes Sociais de Apoio dos usuários da \acrshort{ABS} de acordo com a literatura?

O tema despertou-me o interesse a partir das vivências durante a disciplina Internato I onde enquanto desenvolvia as atividades na \acrshort{ESF} do município de Pacoti-CE, pude observar o importante papel do enfermeiro na comunidade, formando suas redes junto aos \acrshort{ACS}, à população e funcionários da Atenção Terciária, além de gestores da Secretaria Municipal de Saúde para ampliar o acesso dos pacientes aos serviços ou servindo de apoio e referência nas situações que interferiam em seu estado de saúde-doença. Ademais, a participação no grupo de pesquisa Políticas, Saberes e Práticas em Enfermagem e Saúde Coletiva promoveu meu primeiro contato com a temática e despertou-me as primeiras leituras e questionamentos. 

A luz desse conhecimento, poderá se visualizar uma atividade exercida pela enfermagem dentro da \acrshort{ABS} que não é reconhecida ou instituída de forma clara e orientada, mas que permite as trocas entre os participantes desse retículo, as forças dessas ligações e em como elas auxiliam o acesso aos serviços de saúde e destes à comunidade, fortificando assim, o cuidado integral dos sujeitos e das coletividades.   
