\chapter{Introdução}
\label{cap:introducao}

O \acrlong{SUS} (\acrshort{SUS}) é composto de níveis de atenção (primário, secundário e terciário) em que geralmente a Atenção Básica deve ser o contato preferencial dos usuários, porta de entrada e centro de comunicação com toda a Rede de Atenção à Saúde. \cite{ministerio2012politica}.

O enfermeiro está presente em todos os  níveis de atenção desenvolvendo atividades tanto assistenciais quanto gerenciais. Mesmo com a sua presença em ambientes tão diversos é na \acrlong{AB} (\acrshort{AB}) que suas habilidades tem potencial para se desenvolver com maior autonomia. 

Em um estudo de \cite{kalinowski2013autonomia} os enfermeiros perceberam que tinham autonomia profissional quando surgiam situações com a possibilidade de tomar decisões no serviço de saúde e também no seu processo de trabalho, utilizando dispositivos indispensáveis como competência, responsabilidade, respeito e reconhecimento na equipe interdisciplinar.

Essa autonomia decorre não apenas de uma postura profissional, o saber-ser do enfermeiro, mas também da própria dinâmica da \acrshort{AB}, que enfatiza uma assistência multidisciplinar ao paciente e às comunidades. 

No mesmo estudo, comprovou-se que a enfermeira exerce no seu campo de trabalho diferentes atividades, cabendo à ela, reorganizar tanto o seu processo de trabalho, quanto  o da sua equipe, criando assim, maior visibilidade da sua prática. 

Este profissional é muitas vezes o articulador da equipe de saúde na AB, pois com suas competências gerenciais, tem no trabalho em equipe um caminho para a comunicação e contato entre os diferentes profissionais. Para \cite{rocha2013avaliaccao}, graças às atribuições do enfermeiro na \acrlong{ESF} (\acrshort{ESF}), este vem assumindo cada vez mais o papel de gerente das unidades. Isto exige dele um olhar mais amplo sobre o trabalho, despertando o reconhecimento dos outros profissionais como articulador, facilitador e mediador de muitas ações. 

A \acrshort{ESF} é uma estratégia que visa reorganizar a \acrshort{AB} no país de acordo com os preceitos do \acrshort{SUS}. Ela tem caráter de expansão, qualificação e consolidação da \acrshort{AB}, pois favorece reorientação do processo de trabalho com potencial de aprofundar princípios, diretrizes e fundamentos da \acrshort{AB}, ampliando sua resolutividade e impacto na situação de saúde das pessoas e coletividades, além de proporcionar uma importante relação de custo-efetividade. \cite{ministerio2012politica}.

Para isso ela apresenta equipes multidisciplinares, que minimamente são formadas por por um médico generalista ou especialista em Saúde da Família, um enfermeiro generalista ou especialista em Saúde da Família, auxiliar ou técnico de enfermagem e agentes comunitários de saúde, podendo-se a esta acrescentar os profissionais de saúde bucal. \cite{ministerio2012politica}.

Considerando-se as diferentes realidades de cada local, as diversas \acrlong{UBS} (\acrshort{UBS}) que compõe a \acrshort{AB} precisam de uma articulação entre seus profissionais e os demais componentes do sistema de saúde, além de articulação também com as comunidades na qual estão inseridas de forma a oferecer assistência tendo como base os princípios da universalidade, da acessibilidade, do vínculo, da continuidade do cuidado, da integralidade da atenção, da responsabilização, da humanização, da equidade e da participação social e um processo de comunicação que vá contribuindo para autonomização e protagonismo de sujeitos nos processos de tomada de decisão.

Assim, a comunicação é uma dos principais componentes para o funcionamento de qualquer grupo. É a competência interpessoal capaz de decodificar as diversas formas de expressão humana capaz de ampliar as relações num dado território. \cite{rocha2013avaliaccao}.

Neste processo de comunicação, tem-se a formação de redes sociais. O conceito de Redes Sociais não é novo e não é restrito, nem restrito. Para \cite{marteleto2010redes} É um conceito onipresente atualmente em diversos espaços e parece servir a dois propósitos: caracterizar o espaço em que a comunicação ocorre no mundo globalizado de hoje onde se tem a produção de formas diferentes de ações coletivas, expressão de identidades, conhecimentos informação e cultura; e para apontar as mudanças no modo de se comunicar e passar adiante a informação.

Assim sendo, \cite{marteleto2010redes} também coloca que as informações e as redes sociais são dois conceitos que se encontram e que permeiam diferentes domínios de conhecimento, mídias, campos sociais ou comunidades profissionais. A troca de informações, seu uso e apropriação vão depender de como as pessoas e grupos envolvidos no processo de comunicação se associam.

Uma proposta para a análise dessas relações que se formam está na \acrlong{ARS} (\acrshort{ARS}) que é uma ferramenta que nos permite conhecer as interações entre qualquer classe de indivíduos, partindo preferencialmente de dados qualitativos do que quantitativos. \cite{alejandro2005manual}.

Na saúde, a \acrlong{ARS} tem como foco a compreensão das relações entre os atores, ou seja, das relações entre os profissionais de diferentes categorias que participam do processo de comunicação, durante o cuidado prestado aos pacientes \cite{antonio2013}.
 
Consiste em uma ramo do campo de estudo das relações interorganizacionais que pode monitorar como ocorrem as trocas presentes na produção do serviço de saúde, observando de que forma a localização dos atores envolvidos se relaciona com poder e influência. \cite{bittencourt2009rede}.

Nesse contexto, os pacientes das várias linhas de cuidado são beneficiados com o bom funcionamento da comunicação, cooperação e do vínculo dos profissionais da unidade na qual são assistidos e da unidade com os outros serviços de saúde disponíveis. 

Segundo \cite{cecilio2003integralidade}, a cocepção de linha de cuidado ilustra a produção da saúde de forma sistêmica, partindo das redes macro e microinstitucionais, com processos dinâmicos, onde se tem a imagem da linha de produção direcionada ao fluxo de assistência àquele que dela irá se beneficiar de acordo com suas necessidades.

A linha de cuidado pode ser abordada nas perspectivas de macro e micropolítica. Na micropolítica, tem-se o encontro entre o usuário e o profissional e nela, torna-se essencial que a assistência passe de procedimentos fragmentados a ações de responsabilização, vinculação e cuidado, possibilitando assim projetos terapêuticos singularizados. \cite{malta2010percurso}.

Quanto a macropolítica, tem-se as relações entre os gestores e profissionais envolvidos no cuidado, fomentando corresponsabilização de forma a garantir apoio para as ações de cuidado. \cite{malta2010percurso}.

Dentre os que mais buscam o atendimento a nível primário, encontramos os pacientes com \acrlong{HAS} (\acrshort{HAS}) e \acrlong{DM} (\acrshort{DM}). Estes pacientes muitas vezes requerem um atendimento diferenciado envolvendo uma diversidade de ações de cuidado que são implementadas por diferentes profissionais. Isto se deve ao aspecto crônico dessas patologias e aos fatores e determinações  a elas associados. 

Existe também a possibilidade do desenvolvimento de comorbidades associadas ao estado sistêmico em que o indivíduo se encontra, tudo isso dependendo do estilo de vida e seus modos de compreender a produção da saúde desses pacientes, envolvendo ainda seu contexto familiar e social. 

Por exemplo, em um estudo desenvolvido com idosos hipertensos de um município do Paraná, \cite{ferrari2014motivos}, identificaram que as principais queixas que levavam esses pacientes a buscarem os serviços da \acrshort{UBS} eram relacionados a doenças endócrinas, nutricionais e metabólicas. Consistiam em problemas como excesso de peso, elevação de níveis glicêmicos e hipercolesterolemia. Demandas como essas podem envolver toda a equipe em um cuidado que requer diversas abordagens profissionais. 

Este contexto colabora para tornar o acompanhamento da \acrshort{HAS} e \acrshort{DM} na atenção primária um fator que pode evitar o surgimento e progressão de complicações. Isto faz com que se reduzam as internações hospitalares e mortalidade relacionada a esses agravos \cite{da2012associaccao}.

A \acrshort{AB} para esses pacientes funciona então não só como porta de entrada, mas como importante território de matriciamento para acompanhamento e suporte à nova realidade a que esses pacientes devem se adaptar. Além disso, cuidados especializados devem ser providenciados através do sistema de referência, encaminhando o paciente a outros serviços dentro do sistema de saúde. 

Uma vez que o enfermeiro se localiza em uma posição primordial para articular ações que promovam o acesso aos serviços de saúde e a continuidade do cuidado, contemplando assim a integralidade da assistência, através da comunicação e da ativação de sua própria rede social ou interpessoal, questiona-se: como se configura a Rede Social para a linha de cuidado a pacientes hipertensos e diabéticos de uma enfermeira da \acrshort{ESF} de um município de pequeno porte?

\section{Justificativa}
A escolha pelo tema teve origem com a vivência como acadêmica da disciplina Internato I durante o módulo de Atenção Básica na \acrshort{ESF} em Pacoti-Ce. Como este sendo um município de pequeno porte, foi-me possível observar a rede de assistência com mais detalhes e percebi que em diversas situações a enfermeira da equipe intermediava ações de cuidado e assistência dos profissionais da unidade e com as demais instituições da rede municipal de saúde, além de acionar a participação de indivíduos da comunidade que facilitavam o acesso a variados equipamentos sociais que eram utilizados para o desenvolvimento de atividades de educação em saúde na comunidade. 

Percebi também a constante busca pela população com \acrshort{HAS} e \acrshort{DM}, que prioritariamente buscava a consulta médica devido a prescrição de algumas medicações, mas que quando passavam pela consulta de enfermagem, relatavam outras demandas. Em muitas dessas situações, a enfermeira recorria aos próprios contatos no hospital municipal, na Secretaria de Saúde, além do contato direto com os \acrlong{ASCs} (\acrshort{ASCs}) e os demais profissionais da unidade. Essas situações geralmente se relacionavam a falta de acesso a um serviço ou a demora na resolução de uma demanda. 

Contou também para a escolha do tema o interesse em novas metodologias para análises no campo da Saúde Coletiva, sobretudo ligadas ao processo de trabalho do enfermeiro. Essas discussões foram suscitadas durante algumas reuniões do grupo de pesquisa Políticas, Saberes e Práticas em Enfermagem e Saúde Coletiva da \acrlong{UECE} (\acrshort{UECE}).  

\section{Relevância}
Baseado  no exposto é relevante mapear esta forma de enfrentamento de múltiplas situações que incidem sobre o processo de trabalho do enfermeiro e assim compreender melhor os elementos que permitem a este profissional configurar, participar e/ou liderar, explicita ou tacitamente as redes sociais visando responder às demandas e necessidades cotidianas dos usuários hipertensos e diabéticos. 

Além disso, o presente estudo serve como passo inicial na investigação de como contribuir para a reorganização do processo de trabalho da equipe multidisciplinar da unidade  básica com os outros serviços da rede de forma mais  efetiva, solidária e cooperativa, facilitando o acesso e o cuidado integral dos pacientes dessa linha de cuidado.