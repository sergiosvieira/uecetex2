Dentre os que mais buscam o atendimento a nível primário, encontramos os pacientes com HAS (HAS) e DM (DM). Estes muitas vezes requerem um atendimento diferenciado envolvendo uma diversidade de ações de cuidado que são implementadas por diferentes profissionais. A AB  funciona então não só como porta de entrada, mas como importante território de matriciamento para acompanhamento e suporte à nova realidade a que esses usuários devem se adaptar. O enfermeiro  localiza-se em uma posição primordial para articular ações que promovam o acesso aos serviços de saúde e a continuidade do cuidado, contemplando assim a integralidade da assistência, através da comunicação e da ativação de sua própria rede social ou interpessoal. A fim de se elucidar como se dá essa articulação e como ela interfere na resolutividade das demandas desses pacientes, objetivou-se nesse estudo, analisar a rede social de uma enfermeira da Estratégia Saúde da Família de Icapuí-Ce. A análise de redes sociais na saúde constitui um campo de grande interesse. Através das redes de relacionamento, por exemplo, comunidades podem buscar melhorias na sua realidade. Trata-se de um estudo de caso de abordagem qualitativa e exploratória realizado em uma Unidade de Atenção Primária à Saúde do município de Icapuí-Ce. Este trabalho compõe-se de um recorte da pesquisa Redes sociais no trabalho de enfermeiros da Atenção Básica: um estudo em municípios do Rio de Janeiro e Ceará. Participaram desse estudo a enfermeira da UBS de Barreira e a partir dela, demais profissionais apontados como componentes da rede social que fossem trabalhadores da saúde com diferentes vínculos com o município ou com sua rede de assistência. A coleta realizou-se em junho de 2015 por meio de  uma entrevista semi-estruturada. A pesquisa respeitou os preceitos éticos estabelecidos pela Resolução 466/2012 do Conselho Nacional de Saúde, tendo parecer favorável do Comitê de Ética em Pesquisa da Universidade Estadual do Ceará (n$^o$ 818.029/2014).  Foi utilizada a técnica de Análise de Redes Sociais (ARS) com o auxílio do software UCINET versão 6.18 e Netdraw. O grafo gerado foi analisado visando identificar quais foram os profissionais mais acessados pela enfermeira cuja rede foi inicialmente analisada, como também a localização da mesma profissional dentro da rede que se construiu a partir dos atores posteriormente citados. Foram utilizadas as medidas de densidade da rede, grau de centralidade, grau de proximidade e intermediação dos nós. Os resultados que surgiram mostram a comunicação bem articulada entre os enfermeiros e os demais profissionais de enfermagem. Mesmo com outras categorias participando da rede e sendo citadas, o desenho da mesma aponta para uma necessidade de envolvimento de outros trabalhadores da saúde no processo. Viu-se também que a Residência Multiprofissional exerce impacto na continuidade do cuidado e no fácil acesso a esses serviços. Conclui-se que no presente estudo emergiram atores que em sua maioria eram enfermeiros ou profissionais da enfermagem, identificando o potencial do enfermeiro para estabelecer a comunicação entre outros trabalhadores da saúde e diferentes setores e instituições de saúde do município.

% Separe as palavras-chave por ponto
\palavraschave{Enfermagem. Redes Sociais. Atenção Primária à Saúde}