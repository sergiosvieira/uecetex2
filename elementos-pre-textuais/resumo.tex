A sífilis é uma enfermidade que persiste até os dias de hoje. Seu tratamento é relativamente simples e seu diagnóstico já é rotina nos serviços de saúde durante o pré-natal. Apesar disso, ela ainda existe em números alarmantes de casos entre gestantes até o fim da gestação, causando Sífilis Congênita, que é a sífilis adquirida pelo feto por transmissão vertical. A sífilis causa diversos danos a gravidez e ao concepto. O diagnóstico de sífilis é feito na Atenção Básica e quando diagnosticado precocemente seu tratamento também se inicia no pré-natal. Entre os obstáculos a resolução desse problema estão a adesão ao tratamento tanto da gestante quanto do seu parceiro, o manejo dos profissionais a estas pacientes e o registro adequado tanto na caderneta da gestante quanto nos sistemas de informação. O enfermeiro encontra-se em uma posição de destaque dentro desse contexto, pois é esse profissional que inicia o pré-natal e é o responsável por muitas das ações que o integram. Com o objetivo de investigar como ocorre o atendimento no pré-natal a essas gestantes e as ações que o enfermeiro utiliza para aumentar a adesão a terapêutica a sífilis realizou-se neste trabalho uma Revisão Integrativa. Para isso foram usados os descritores Enfermagem, Sífilis e Pré-Natal com o booleando AND entre os termos para a busca no portal da Biblioteca Virtual em Saúde (BVS). Foram obtidos estudos com abordagens diversas ao problema da sífilis na gestação. Em sua maioria, os trabalhos faziam uso dos dados disponibilizados nos sistemas de informação. A análise de alguns apontavam para uma satisfatória cobertura das equipes de Saúde da Família nos territórios, mas não mostravam diminuições significativas na incidência de sífilis congênita, denotando falhas na assistência a gestante durante o pré-natal. É necessário não só um maior conhecimento a respeito dos fatores intervenientes dessa assistência, mas também das estratégias dos profissionais para superá-las a fim de promoverem realmente uma consulta resolutiva para um problema ainda tão pertinente e de grave desfecho na saúde materno-infantil. 

% Separe as palavras-chave por ponto
\palavraschave{Enfermagem. Sífilis. Pré-natal.}