A violência obstétrica está relacionada a ações realizadas pelos profissionais da saúde no que diz respeito ao corpo e aos processos reprodutivos da mulher, abusando de ações intervencionistas, medicalização e atos desumanos tornando o processo de parto natural para patológico. Em vista disso, ressalta-se a importância do Centro de Parto Normal (CPN), uma ferramenta de saúde que presta assistência humanizada às gestantes através de boas práticas e uso adequado de tecnologia, seguindo as recomendações da OMS. Neste cenário, é relevante a atuação do enfermeiro obstetra devido à sua formação mais humanizada no atendimento a gestante de risco habitual e ao recém-nascido. Objetivou-se desvelar a violência obstétrica no Centro de Parto Normal, na perspectiva do enfermeiro. Trata-se de uma pesquisa com abordagem qualitativa do tipo exploratória e descritiva, que será realizada com a participação de 6 enfermeiras atuantes no CPN do município de Maracanaú habilitado pelo Ministério de Saúde do Ceará. A pesquisa ocorreu em 2018, com a coleta dos dados entre os meses de agosto a setembro realizada por meio de entrevistas semi-estruturadas, terá como ferramenta organizadora para construção da discussão a análise temática. Os preceitos éticos serão respeitados, regidos pela Resolução n. 466/12 do Conselho Nacional de Saúde.

% Separe as palavras-chave por ponto
\palavraschave{Violência Obstétrica. Centro de Parto Normal. Enfermagem.}