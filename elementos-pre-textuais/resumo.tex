Este estudo trata-se uma Revisão Integrativa que teve como objetivo investigar como ocorre a consulta de enfermagem no pré-natal a gestantes com sífilis, bem como as ações que o enfermeiro desenvolve para estimular a adesão das pacientes ao tratamento. Para isso foram usados os descritores Enfermagem, Sífilis e Pré-Natal com o booleando AND entre os termos para a busca no portal da Biblioteca Virtual em Saúde (BVS). Foram obtidos estudos com abordagens diversas ao problema da sífilis na gestação. Em sua maioria, os trabalhos faziam uso dos dados disponibilizados nos sistemas de informação. A análise de alguns apontavam para uma satisfatória cobertura das equipes de Saúde da Família nos territórios, mas não mostravam diminuições significativas na incidência de sífilis congênita, denotando falhas na assistência a gestante durante o pré-natal. É necessário não só um maior conhecimento a respeito dos fatores intervenientes dessa assistência, mas também das estratégias dos profissionais para superá-las a fim de promoverem realmente uma consulta resolutiva para um problema ainda tão pertinente e de grave desfecho na saúde materno-infantil. 

% Separe as palavras-chave por ponto
\palavraschave{Enfermagem. Sífilis. Pré-natal.}