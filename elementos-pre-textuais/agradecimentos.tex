Primeiramente a Deus que permitiu que tudo isso acontecesse, dando-me forças ao longo dessajornada.

Aos meus pais, pelo amor, incentivo e apoio incondicional.

Obrigada minhas irmãs, que me incentivaram a perseguir esse objetivo e com quem desabafei incontáveis vezes sobre os desafios de se conquistar a excelência na profissão frente a uma sociedade ainda tão discriminatória com a figura da mulher.
Gratidão ao meu esposo Sérgio e a minha filha Margot, luzes da minha vida e bálsamo nos momentos mais difíceis.

A Universidade Estadual do Ceará e ao Curso de Especialização em Enfermagem Obstétrica, seu corpo docente, direção e administração que oportunizaram a janela que hoje vislumbro um horizonte superior, seguindo os passos que ainda na graduação aprendi comecei a trilhar nesta instituição.

Agradeço a todos os professores e professoras por me proporcionar o conhecimento não apenas racional, mas a manifestação do caráter e afetividade da educação no processo de formação profissional, por tanto que se dedicaram a mim, não somente por terem me ensinado, mas por terem me feito aprender. a palavra mestre, nunca fará justiça aos professores dedicados aos quais sem nominar terão os meus eternos agradecimentos.
Gratidão a todos os profissionais que me guiaram nos campos de estágio e com os quais pude viver a saúde da mulher em seus vários contextos.

Obrigada à minha preceptora, enfermeira Robeísa, pela atenção, humanização, confiança e competência demonstrados ao longo do meu aprendizado.
Gratidão as amigas e companheiras do curso Clara, Vitória, Bárbara, Débora, Sâmela e Renata, pelo companheirismo e aprendizados em conjunto.

Gratidão a profa. Dra. Maria Rocineide Ferreira da Silva, grande mestra e pessoa, por me acompanhar em mais um trabalho e por continuar me instruindo pelos caminhos da Enfermagem.

Também um agradecimento especial às enfermeiras Ms. Katherine Jerônimo Lima e Ms. Leidy Dayane Paiva de Abreu por aceitarem o convite para a avaliação e enriquecimento deste trabalho. 

Finalmente, sou grata a todas as mulheres e suas famílias que me permitiram participar de um momento tão especial como o nascimento de seus filhos e filhas, iniciando-me nessa jornada.
