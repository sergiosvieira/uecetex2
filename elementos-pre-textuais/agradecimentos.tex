Agradeço a Deus por estar dentro do meu coração, mostrando-me meu caminho e me dando discernimento e coragem para ajustar as rotas da vida. 

Aos meus pais Luís Carlos e Maria Lúcia, por sempre acreditarem em mim e terem tido paciência em todos os meus momentos de incerteza. Por toda abdicação, dedicação à família e amor por toda uma vida.

Ao meu esposo, Antônio Sérgio de Sousa Vieira, por todo amor, carinho, respeito e companheirismo. Por acreditar no meu potencial, dando-me forças e ajudando-me em todos os momentos que preciso inclusive, na execução deste trabalho. Que nosso caminho continue se entrelaçando para além dessa vida.

Aos meus irmãos Alan, Tatiana e Mariana pelos momentos em família e pelo afeto. Todo o amor fraternal.

À Prof. Dra. Maria Rocineide Ferreira da Silva, minha orientadora e fonte de inspiração. A quem vejo como um exemplo de pessoa e de profissional, dedicada, competente, responsável e sobretudo, humada. Foi uma honra tê-la como minha orientadora e mestre. Agradeço por todo o carinho,compreensão, paciência e empatia.

À Prof.a Dra. Dafne Paiva Rodrigues, tutora e professora. Pessoa competente, responsável e dedicada ao ensino da Enfermagem.

À Profa. Dra. Lucilane Maria Sales da Silva e Ms. Thayza Miranda Pereira por terem aceitado o convite para compor a minha banca de monografia, muito tendo contribuído para a elaboração e aperfeiçoamento deste trabalho.

A todos os professores do curso de Enfermagem que tanto contribuíram para a minha formação acadêmica.

Aos meus amigos da UECE por todos os momentos compartilhados, de estudos e de diversão. Em especial a Jordana Moreira, Teresa Cristina, Carlos Bruno e Delani Fiusa. Desejo tê-los sempre por perto.

A todos do PET-Enfermagem e do LAPRACS, aprendi e cresci muito com vocês.

A todos os campos de estágio em que tive a oportunidade de realizar o internato, foram fundamentais para minha formação prática.

À Secretaria Municipal de Saúde e as instituições de saúde de Icapuí, representadas nas figuras de seus profissionais que autorizaram e colaboraram para a elaboração desse trabalho.

À Universidade Estadual do Ceará pela oportunidade de me graduar em um curso gratuito e de qualidade.
