Among those who most seek care at the primary level, we find patients with hypertension and diabetes. They often require a differentiated service involving a diversity of care actions that are implemented by different professionals. The primary care so it works not only as a gateway, but as an important territory of matricial for monitoring and support to the new reality to which they must adapt. The nurse is located in a prime position to articulate actions to promote access to health services and continuity of care, thus covering comprehensive care through communication and activation of their own social or interpersonal network. In order to elucidate how is this articulation and how it interferes in solving the demands of these patients, the aim of this study was to analyze the social network of a Nurse Health Strategy on Icapui-Ce. It used the technique of Social Network Analysis with the aid of software version UCINET 6:18 and Netdraw. They analyzed the network density, degree of centrality, degree of proximity and intermediation of us. It was concluded that in this study emerged actors who mostly were nurses or nursing professionals, identifying the nurse's potential to establish communication between other healthcare workers and different sectors and municipal health institutions.
% Separe as Keywords por ponto
\keywords{Social Network Analysis. Nursing. Primary Health}