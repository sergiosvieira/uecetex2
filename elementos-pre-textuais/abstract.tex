Syphilis is a disease that persists to this day. Its treatment is relatively simple and its diagnosis is already routine in health services during prenatal care. Despite this, it still exists in alarming numbers of cases among pregnant women until the end of gestation, causing Congenital Syphilis, which is syphilis acquired by the fetus by vertical transmission. Syphilis causes various damages to pregnancy and the concept. The diagnosis of syphilis is made in Primary Care and when diagnosed early, its treatment also begins in prenatal care. Among the obstacles to solve this problem are adherence to the treatment of both the pregnant woman and her partner, the management of the professionals to these patients and the adequate registration both in the pregnant woman's book and in the information systems. The nurse is in a position of prominence within this context, since it is this professional who initiates prenatal care and is responsible for many of the actions that integrate it. In order to investigate how prenatal care for these pregnant women occurs and the actions that the nurse uses to increase adherence to syphilis therapy, an Integrative Review was carried out. For this, the descriptors Nursing, Syphilis and Prenatal with the booleando AND among the terms were searched for in the portal of the Virtual Health Library (VHL). We obtained studies with different approaches to the problem of syphilis during pregnancy. For the most part, the works made use of the data available in the information systems. The analysis of some pointed to a satisfactory coverage of the Family Health teams in the territories, but did not show significant decreases in the incidence of congenital syphilis, denoting faults in the assistance to pregnant women during prenatal care. There is a need not only for greater knowledge about the factors involved in this assistance, but also for the strategies of the professionals to overcome them in order to really promote a resolutive consultation on a problem that is still so relevant and of serious outcome in maternal and child health.

% Separe as Keywords por ponto
\keywords{Nursing. Syphilis. Prenatal.}