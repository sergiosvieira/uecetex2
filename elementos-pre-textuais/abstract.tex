Obstetric violence is related to actions taken by health professionals regarding the body and the reproductive processes of women, abusing interventionist actions, medicalization and inhumane acts, making the process of childbirth natural for pathological. In view of this, the importance of the Normal Childbirth Center (NCC), a health tool that provides humanized assistance to pregnant women through good practices and appropriate use of technology, is underlined, following WHO recommendations. In this scenario, the performance of obstetrician nurses is relevant because of their more humanized training in attending pregnant women at normal risk and the newborn. The objective of this study was to reveal obstetric violence in the Normal Childbirth Center, from the perspective of nurses. This is a research with a qualitative exploratory and descriptive approach, which will be carried out with the participation of 6 nurses working at the NCC in the municipality of Maracanaú, authorized by the Ministry of Health of Ceará. The research was conducted in 2018, with the collection of data between the months of August to September conducted through semi-structured interviews, has as an organizing tool for the construction of the discussion thematic analysis. The ethical precepts will be respected, governed by Resolution n. 466/12 of the National Health Council. Through the analysis of the obstetrical nurses' discourses, we constructed three categories of results: the obstetrical nurse's perception of obstetric violence; practice of violence in the daily life of the NCC and possibilities of coping with obstetric violence in the NCC. The obstetrical nurses working in the NCC revealed their perceptions about obstetric violence with propriety, however we still have some limitations due to the new look at the care of these professionals in the NCC.

% Separe as Keywords por ponto
\keywords{Obstetric Violence. Normal Childbirth Center. Nursing.}