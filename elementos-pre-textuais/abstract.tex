This study is an Integrative Review that aimed to investigate how prenatal nursing consultation occurs in pregnant women with syphilis, as well as the actions that the nurse develops to stimulate patients adherence to treatment. For this, the descriptors Nursing, Syphilis and Prenatal with the booleando AND among the terms were searched for in the portal of the Virtual Health Library (VHL). We obtained studies with different approaches to the problem of syphilis during pregnancy. For the most part, the works made use of the data available in the information systems. The analysis of some pointed to a satisfactory coverage of the Family Health teams in the territories, but did not show significant decreases in the incidence of congenital syphilis, denoting faults in the assistance to pregnant women during prenatal care. There is a need not only for greater knowledge about the factors involved in this assistance, but also for the strategies of the professionals to overcome them in order to really promote a resolutive consultation on a problem that is still so relevant and of serious outcome in maternal and child health.

% Separe as Keywords por ponto
\keywords{Nursing. Syphilis. Prenatal.}