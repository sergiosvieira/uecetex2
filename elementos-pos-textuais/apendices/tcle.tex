\apendice{Termo de Consentimento Livre e Esclarecido}
\label{ap:tcle}

Bom dia/ boa tarde. Prezado (a) enfermeiro(a), meu nome é Sâmela Abreu e Santiago, aluna da especialização em enfermagem obstetrícia da Universidade Estadual do Ceará e estou desenvolvendo, sob orientação do prof. Dr. Antonio Rodrigues Ferreira Júnior, uma pesquisa intitulada  \textbf{``Percepções do enfermeiro sobre a violência obstétrica no centro de parto normal''}, cujo objetivo principal é \textbf{Desvelar a violência obstétrica no Centro de Parto Normal, na perspectiva do enfermeiro}. Para alcançar esse objetivo estamos convidando para participar da pesquisa os enfermeiros do Centros de Parto Normal do município de Maracanaú, habilitado pelo Ministério da Saúde no Ceará. Caso aceite, sua participação consistirá em responder a uma entrevista individual cujo propósito é conhecer um pouco de suas práticas no Centro de Parto Normal. Ressaltamos que esta pesquisa não tem o objetivo de avaliar seus conhecimentos, que suas respostas não serão identificadas e que os resultados obtidos contribuirão para discussão sobre a assistência prestada nos Centros de Parto Normal. Não haverá nenhuma forma de reembolso de dinheiro, já que com a participação na pesquisa você não vai ter nenhum gasto. As informações fornecidas por você serão utilizadas somente para fins dessa pesquisa e em publicações científicas que dela resultarem e sempre será preservado o seu anonimato. Quanto aos riscos, caso haja algum tipo de constrangimento ou desconforto, você poderá nos comunicar e tomaremos medidas imediatas (abortar a entrevista) e mediatas (retirar seus dados da pesquisa, se desejar, e, se necessário for, ajudar-lhe com o encaminhamento a um serviço de atenção psicológica). Entregarei uma cópia deste termo ao Sr. (a). Gostaríamos de fazer algumas perguntas e toda a informação que o Sr. (a) nos der será utilizada somente para esta pesquisa. Como é difícil escrever tudo o que for falado, gostaríamos de gravar esta conversa. Somente os pesquisadores envolvidos terão acesso à gravação. Todas as suas respostas serão confidenciais e seu nome não será registrado em nenhum lugar. Caso queira, encaminharemos também a transcrição de toda a entrevista via email ou correios. 

Sua colaboração é muito importante para a pesquisa e para a construção de conhecimento de uma área especial, como é o caso da atenção obstétrica no Centro de Parto Normal. 

Entretanto, a decisão em participar deve ser sua. Se você não concordar em participar, ou quiser desistir a qualquer momento, retirando as suas informações, isso não lhe trará nenhum prejuízo ou constrangimento. Para isso, basta você entrar em contato comigo através do telefone e e-mail constantes abaixo. O Comitê de Ética em Pesquisa (CEP) é formado de um grupo de profissionais de diversas áreas, cuja função é avaliar as pesquisas com seres humanos. O CEP foi criado para defender os interesses dos participantes da pesquisa e qualquer dúvida ética poderá entrar em contato.

\bigskip
\bigskip
\noindent
\textbf{Sâmela Abreu e Santiago (pesquisadora)}
Rua José Ivo, 700, Cs. 16, Messejana, Fortaleza-CE. CEP: 60871-041 
Telefone: (85) 99641-4216 E-mail: samela.abreu21@gmail.com 

\bigskip
\bigskip
\noindent
\textbf{Comitê de Ética em Pesquisa da Universidade Estadual do Ceará}
Rua: Silas Munguba, 1700, Itaperi, Fortaleza-CE. CEP 60714-903 
Telefone: (85) 3101-9890 E-mail: cep@uece.br 

\bigskip
\bigskip
\noindent
Eu, \rule[0cm]{5in}{0.1pt}, após tomar conhecimento da forma como será realizada a pesquisa, aceito, de forma livre e esclarecida, participar da mesma.  

\bigskip
\bigskip
\noindent
Fortaleza - CE, \rule[0cm]{0.5in}{0.1pt}/\rule[0cm]{0.5in}{0.1pt}/2018.  

\bigskip
\bigskip
\noindent
\begin{center}
\rule[0cm]{5in}{0.1pt}
\end{center}
\begin{center}
Participante
\end{center}

\bigskip
\bigskip
\noindent
\begin{center}
\rule[0cm]{5in}{0.1pt}
\end{center}
\begin{center}
Pesquisadora
\end{center}

